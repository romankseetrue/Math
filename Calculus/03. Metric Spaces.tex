\documentclass{article}

\usepackage[utf8]{inputenc}
\usepackage[russian, french]{babel}
\usepackage{amsthm}
\usepackage{amsmath}
\usepackage{mathtools}
\usepackage{amssymb}
\usepackage[symbol]{footmisc}
\usepackage{graphicx}
\usepackage{tikz}

\graphicspath{ {./images/} }

\title{Метрические пространства}
\author{Роман Кушниренко}

\makeatletter
\newcommand*{\rom}[1]{\expandafter\@slowromancap\romannumeral #1@}
\makeatother

\renewcommand{\thefootnote}{\fnsymbol{footnote}}
\renewcommand*{\dotFFN}{}

\newtheorem{theorem}{Теорема}[section]
\newtheorem*{lemma}{Лемма}
\newtheorem{claim}{Утверждение}[section]
\newtheorem{example}{Пример}
\newtheorem{definition}{Определение}[section]
\newtheorem*{consequence}{Следствие}

\begin{document}

	\selectlanguage{russian}

	\maketitle

	\section{Понятие метрического пространства}

Одной из важных операций анализа является предельный переход. В основе этой операции лежит тот факт, что на числовой прямой определено расстояние от одной точки до другой. Многие фундаментальные факты анализа не связаны с алгебраической природой вещественных чисел, а опираются лишь на понятие расстояния. Обобщая представление о вещественных числах как о множестве, в котором введено расстояние между элементами, мы приходим к понятию метрического пространства \(-\) одному из важнейших понятий современной математики.

\begin{definition}
Метрическим пространством называется пара \((X, \rho)\), состоящая из некоторого множества \(X\) и функции расстояния \(\rho : X \times X \to \mathbb{R^{+}}\), подчиненной следующим трем аксиомам:
\begin{enumerate}
    \item \(\rho(x, y) = 0\) тогда и только тогда, когда \(x = y\),
    \item \(\rho(x, y) = \rho(y, x)\) (аксиома симметрии),
    \item \(\rho(x, z) \leq \rho(x, y) + \rho(y, z)\) (аксиома треугольника).
\end{enumerate}
\end{definition}

В случаях, когда недоразумения исключены, мы будем зачастую обозначать метрическое пространство тем же символом, что и сам «запас точек» \(X\). \newline

Самыми важными примерами метрических пространств с нашей точки зрения служат пространства \(\mathbb{R}^n\), особенно прямая \(\mathbb{R}^1\) и плоскость \(\mathbb{R}^2\).

\begin{definition}
Множество упорядоченных наборов из \(n\) вещественных чисел \(x = (x_1, ..., x_n)\) с расстоянием
\[
\rho(x, y) = \sqrt{\sum_{k = 1}^{n}(y_k - x_k)^2}
\]
называется пространством \(\mathbb{R}^n\).
\end{definition}

\begin{theorem}
\(\mathbb{R}^n\) \(-\) метрическое пространство.
\end{theorem}

\begin{proof}
Справедливость аксиом \textit{1} и \textit{2} для \(\mathbb{R}^n\) очевидна. Покажем, что в \(\mathbb{R}^n\) выполнена и аксиома треугольника.

Пусть \(x = (x_1, ..., x_n)\), \(y = (y_1, ..., y_n)\), \(z = (z_1, ..., z_n)\); тогда аксиома треугольника записывается в виде
\[
\sqrt{\sum_{k = 1}^{n}(z_k - x_k)^2} \leq \sqrt{\sum_{k = 1}^{n}(y_k - x_k)^2} + \sqrt{\sum_{k = 1}^{n}(z_k - y_k)^2}.
\]
Полагая \(y_k - x_k = a_k\), \(z_k - y_k = b_k\), получаем \(z_k - x_k = a_k + b_k\), а неравенство принимает при этом вид
\[
\sqrt{\sum_{k = 1}^{n}(a_k + b_k)^2} \leq \sqrt{\sum_{k = 1}^{n}a_k^2} + \sqrt{\sum_{k = 1}^{n}b_k^2}.
\]
Но это неравенство сразу следует из известного неравенства Коши-Буняковского
\footnote{
Неравенство Коши-Буняковского вытекает из тождества
\[
\left(\sum_{k = 1}^{n}a_k{b_k}\right)^2 = \sum_{k = 1}^{n}a_k^2 \cdot \sum_{k = 1}^{n}b_k^2 - {1 \over 2}\sum_{i = 1}^{n}\sum_{j = 1}^{n}(a_ib_j - b_ia_j)^2,
\]
которое проверяется непосредственно.
}
\[
\left(\sum_{k = 1}^{n}a_k{b_k}\right)^2 \leq \sum_{k = 1}^{n}a_k^2 \cdot \sum_{k = 1}^{n}b_k^2.
\]
Действительно, в силу этого неравенства имеем
\[
\sum_{k = 1}^{n}(a_k + b_k)^2 = \sum_{k = 1}^{n}a_k^2 + 2 \sum_{k = 1}^{n}a_k{b_k} + \sum_{k = 1}^{n}b_k^2 \leq \left(\sqrt{\sum_{k = 1}^{n}a_k^2} + \sqrt{\sum_{k = 1}^{n}b_k^2}\right)^2,
\]
что и требовалось доказать.
\end{proof}

Неограниченное количество дальнейших примеров дает следующий прием. Пусть \((X, \rho)\) \(-\) метрическое пространство и \(M \subset X\). Тогда \(M\) с той же функцией \(\rho\), которую мы считаем теперь определенной на \(M \times M\), тоже представляет собой метрическое пространство; оно называется подпространством пространства \(X\).

Таким образом, каждое подмножество пространства \(\mathbb{R}^n\) \(-\) метрическое пространство.

\section{Открытые и замкнутые множества}

\begin{definition}
Открытым шаром \(B(x_0, r)\) в метрическом пространстве \(X\) мы будем называть совокупность точек \(x \in X\), удовлетворяющих условию
\[
\rho(x_0, x) < r.
\]
Точка \(x_0\) называется центром этого шара, а число \(r\) \(-\) его радиусом.
\end{definition}

\begin{definition}
Открытый шар радиуса \(\varepsilon\) с центром \(x_0\) мы будем называть также \(\varepsilon\)-окрестностью точки \(x_0\) и обозначать символом \(O_\varepsilon(x_0)\).
\end{definition}

\begin{definition}
Точка \(x\) называется внутренней точкой множества \(M\), если существует окрестность \(O_\varepsilon(x)\) этой точки, целиком содержащаяся в \(M\).
\end{definition}

\begin{definition}
Множество, все точки которого внутренние, называется открытым.
\end{definition}

\begin{theorem}
В любом метрическом пространстве открытый шар \(B(x_0, r)\) есть открытое множество.
\end{theorem}

\begin{proof}
Действительно, если \(x \in B(x_0, r)\), то \(\rho(x_0, x) < r\). Положим \(\varepsilon = r - \rho(x_0, x)\). Тогда \(B(x, \varepsilon) \subset B(x_0, r)\).
\end{proof}

\begin{consequence}
Всякая окрестность является открытым множеством.
\end{consequence}

\begin{definition}
Точка \(x\) называется предельной точкой множества \(M\), если каждая окрестность точки \(x\) содержит точку \(y \neq x\), такую, что \(y \in M\).
\end{definition}

\begin{theorem}
Если \(x\) \(-\) предельная точка множества \(M\), то любая окрестность точки \(x\) содержит бесконечно много точек множества \(M\).
\end{theorem}

\begin{proof}
Предположим, что существует окрестность \(O_\delta(x)\), содержащая только конечное число точек множества \(M\). Пусть \(y_1, ..., y_n\) \(-\) те точки множества \(O_\delta(x) \cap M\), которые не совпадают с \(x\). Положим
\[
\varepsilon = \min_{1 \leq m \leq n}{\rho(x, y_m)}.
\]
Ясно, что минимум конечного множества положительных чисел \(-\) положительное число, так что \(\varepsilon > 0\).

Окрестность \(O_\varepsilon(x)\) не содержит ни одной точки \(y \in M\), такой, что \(y \neq x\), поэтому \(x\) не является предельной точкой множества \(M\). Это противоречие и доказывает теорему.
\end{proof}

\begin{consequence}
Конечное множество не имеет предельных точек.
\end{consequence}

\begin{definition}
Множество \(M \subset X\) называется ограниченным, если оно содержится целиком в некотором шаре.
\end{definition}

Теперь мы можем сформулировать и доказать еще одну лемму, касающуюся построенной нами системы вещественных чисел \(\mathbb{R}\).

\begin{lemma}[Больцано, Вейерштрасс]
Всякое бесконечное ограниченное числовое множество имеет по крайней мере одну предельную точку.
\end{lemma}

\begin{proof}
Пусть \(M\) \(-\) данное подмножество \(\mathbb{R}\). Из определения ограниченности множества \(M\) следует, что \(M\) содержится в некотором отрезке \([a, b] = I \subset \mathbb{R}\). Покажем, что по крайней мере одна из точек этого отрезка является предельной для \(M\).

Если бы это было не так, то каждая точка \(x \in I\) имела бы окрестность \(O_\varepsilon(x)\), в которой либо вообще нет точек множества \(M\), либо их там конечное число. Совокупность \(\{O_\varepsilon(x)\}\) таких окрестностей, построенных для каждой точки \(x \in I\), образует открытое покрытие отрезка \(I\), из которого по лемме Гейне-Бореля можно извлечь конечную подсистему \(\{O_{\varepsilon_1}(x_1), ..., O_{\varepsilon_n}(x_n)\}\), покрывающую отрезок \(I\). Но, поскольку \(M \subset I\), эта же подсистема покрывает все множество \(M\). Однако в каждом интервале \(O_{\varepsilon_i}(x_i)\) только конечное число точек множества \(M\), значит, и в их объединении тоже конечное число точек \(M\), т. е. \(M\) \(-\) конечное множество. Полученное противоречие завершает доказательство.
\end{proof}

\begin{definition}
Множество \(M \subset X\) называется замкнутым, если оно содержит все свои предельные точки.
\end{definition}

Каково бы ни было метрическое пространство \(X\), пустое множество \(\O\) и всё \(X\) замкнуты.

Всякое множество, состоящее из конечного числа точек, замкнуто.

\begin{definition}
Замкнутым шаром \(B[x_0, r]\) в метрическом пространстве \(X\) мы будем называть совокупность точек \(x \in X\), удовлетворяющих условию
\[
\rho(x_0, x) \leq r.
\]
Точка \(x_0\) называется центром этого шара, а число \(r\) \(-\) его радиусом.
\end{definition}

\begin{theorem}
В любом метрическом пространстве замкнутый шар \(B[x_0, r]\) есть замкнутое множество.
\end{theorem}

\begin{proof}
\(\forall x \notin B[x_0, r]\) \(\exists \varepsilon = \rho(x_0, x) - r > 0 :\)
\[
O_\varepsilon(x) \cap B[x_0, r] = \O.
\]
Следовательно, точка \(x \notin B[x_0, r]\) не может быть предельной для \(B[x_0, r]\).
\end{proof}

\begin{definition}
Множество всех предельных точек данного множества \(M\) называют производным множеством множества \(M\) и обозначают \(M'\).
\end{definition}

\begin{definition}
Множество, образованное присоединением к множеству \(M\) его производного множества \(M'\), называют замыканием множества \(M\) и обозначают \(\overline{M}\).
\end{definition}

\begin{theorem}
Каково бы ни было множество \(M \subset X\), его производное множество \(M'\) замкнуто.
\end{theorem}

\begin{proof}
Пусть \(x\) \(-\) предельная точка для \(M'\). Это означает, что
\[
\forall \varepsilon > 0 \hspace{1mm} \exists x' \neq x : x' \in O_\varepsilon(x) \cap M'.
\]

Положим \(\delta = \min(\rho(x, x'), \varepsilon - \rho(x, x'))\) и рассмотрим окрестность \(O_\delta(x') \subset O_\varepsilon(x)\). Поскольку \(x' \in M'\) \(-\) предельная точка множества \(M\), то в этой окрестности содержится точка \(x'' \in M : x'' \neq x'\). Более того, \(\delta \leq \rho(x, x') \leq \rho(x, x'') + \rho(x'', x') < \rho(x, x'') + \delta\) \(\Leftrightarrow\) \(\rho(x, x'') > 0\) и \(x'' \neq x\).

Таким образом, мы нашли точку \(x'' \in O_\varepsilon(x) \cap M : x'' \neq x\). Следовательно, \(x\) \(-\) предельная точка множества \(M\) \(\Leftrightarrow\) \(x \in M'\).
\end{proof}

Подобным образом доказывается

\begin{theorem}
Каково бы ни было множество \(M \subset X\), его замыкание \(\overline{M}\) замкнуто.
\end{theorem}

\begin{definition}
Дополнением множества \(M \subset X\) называется множество \(M^c = X \setminus M\).
\end{definition}

\begin{theorem}
Пусть \(\{M_\alpha\}\) \(-\) (конечное или бесконечное) семейство множеств \(M_\alpha\). Тогда
\[
\left(\bigcup\limits_{\alpha} M_\alpha\right)^c = \bigcap\limits_{\alpha}M_\alpha^c.
\]
\end{theorem}

\begin{proof}
Пусть \(A\) и \(B\) \(-\) множества, стоящие в искомом равенстве соответственно слева и справа.

Если \(x \in A\), то \(x \notin \bigcup\limits_{\alpha} M_\alpha\), значит, \(x \notin M_\alpha\) при всех \(\alpha\), поэтому \(x \in M_\alpha^c\) при всех \(\alpha\), так что \(x \in B\). Таким образом, \(A \subset B\).

Обратно, если \(x \in B\), то \(x \in M_\alpha^c\) при всех \(\alpha\), значит, \(x \notin M_\alpha\) при всех \(\alpha\), поэтому \(x \notin \bigcup\limits_{\alpha} M_\alpha\), так что \(x \in A\). Таким образом, \(B \subset A\).

Следовательно, \(A = B\).
\end{proof}

\begin{theorem}
Множество \(M \subset X\) открыто тогда и только тогда, когда его дополнение \(M^c\) замкнуто.
\end{theorem}

\begin{proof}
Сначала предположим, что \(M^c\) замкнуто. Выберем \(x \in M\). Тогда \(x \notin M^c\). Следовательно, \(x\) не может являться предельной точкой множества \(M^c\). Значит, существует окрестность \(O_\varepsilon(x)\), такая, что множество \(M^c \cap O_\varepsilon(x)\) пусто, то есть \(O_\varepsilon(x) \subset M\). Таким образом, \(x\) \(-\) внутренняя точка множества \(M\), и \(M\) открыто.

Теперь предположим, что \(M\) открыто. Пусть \(x\) \(-\) предельная точка множества \(M^c\). Тогда каждая окрестность точки \(x\) содержит некоторую точку множества \(M^c\), так что \(x\) не является внутренней точкой множества \(M\). Поскольку \(M\) открыто, это значит, что \(x \in M^c\). Следовательно, \(M^c\) замкнуто.
\end{proof}

\begin{consequence}
Множество \(M \subset X\) замкнуто тогда и только тогда, когда его дополнение \(M^c\) открыто.
\end{consequence}

\begin{theorem}
Для любого семейства открытых множеств \(\{G_\alpha\}\) множество \(\bigcup\limits_{\alpha}G_\alpha\) открыто.
\end{theorem}

\begin{proof}
Положим \(G = \bigcup\limits_{\alpha}G_\alpha\). Если \(x \in G\), то \(x \in G_\alpha\) при некотором \(\alpha\). Так как \(x\) \(-\) внутренняя точка множества \(G_\alpha\), то \(x\) \(-\) внутренняя точка множества \(G\), и \(G\) открыто. Утверждение доказано.
\end{proof}

\begin{consequence}
Для любого семейства замкнутых множеств \(\{F_\alpha\}\) множество \(\bigcap\limits_{\alpha}F_\alpha\) замкнуто.
\end{consequence}

\begin{theorem}
Для любого конечного семейства открытых множеств \(\{G_i\}_{i=1}^n\) множество \(\bigcap\limits_{i=1}^{n}G_i\) открыто.
\end{theorem}

\begin{proof}
Положим \(H = \bigcap\limits_{i=1}^{n}G_i\). Для любого \(x \in H\) существует окрестность \(O_{\varepsilon_i}(x)\), такая, что \(O_{\varepsilon_i}(x) \subset G_i\) (\(i = 1, ..., n\)). Положим \(\varepsilon = \min(\varepsilon_1, ..., \varepsilon_n)\). Тогда \(O_\varepsilon(x) \subset G_i\) при \(i = 1, ..., n\), так что \(O_\varepsilon(x) \subset H\), и множество \(H\) открыто.
\end{proof}

\begin{consequence}
Для любого конечного семейства замкнутых множеств \(\{F_i\}_{i=1}^n\) множество \(\bigcup\limits_{i=1}^{n}F_i\) замкнуто.
\end{consequence}

В утверждении предыдущей теоремы конечность семейства существенна. Действительно, пусть \(G_n\) \(-\) интервал \(\left(-{1 \over n}, {1 \over n}\right)\) (\(n \in \mathbb{N}\)). Тогда \(G_n\) \(-\) открытое подмножество прямой \(\mathbb{R}^1\). Положим \(G = \bigcap\limits_{n=1}^{\infty}G_n\). Тогда \(G\) состоит из единственной точки (а именно, \(G = \{0\}\)) и поэтому не является открытым подмножеством из \(\mathbb{R}^1\). Таким образом, пересечение бесконечного семейства открытых множеств не обязано быть открытым. \newline

Допустим, что \(G \subset Y \subset X\), где \(X\) \(-\) метрическое пространство. То, что \(G\) \(-\) открытое подмножество пространства \(X\), означает, что с каждой точкой \(z \in G\) связано положительное число \(r\), для которого из условий \(\rho(z, x) < r\), \(x \in X\) следует включение \(x \in G\). Но мы уже заметили, что \(Y\) \(-\) тоже метрическое пространство, так что наше определение с таким же успехом можно отнести к \(Y\). Для полной точности мы будем говорить, что множество \(G\) \textit{открыто относительно} \(Y\), если каждой точке \(z \in G\) отвечает число \(r > 0\), такое, что \(y \in G\), если \(\rho(z, y) < r\) и \(y \in Y\).

Множество может быть открытым относительно \(Y\), не будучи открытым подмножеством пространства \(X\). Например, множество точек \(\{(x, 0) : a < x < b\}\) не есть открытое множество, если рассматривать его как подмножество пространства \(\mathbb{R}^2\), но оно является открытым подмножеством пространства \(\mathbb{R}^1 \times \{0\}\).

Однако между этими понятиями имеется простое соотношение, которое мы сейчас установим.

\begin{theorem}
Пусть \(Y \subset X\). Непустое подмножество \(M\) подпространства \(Y\) открыто относительно \(Y\) тогда и только тогда, когда \(M = G \cap Y\) для некоторого открытого подмножества \(G\) пространства \(X\).
\end{theorem}

\begin{proof}
Допустим, что \(M\) открыто относительно \(Y\). Для каждого \(z \in M\) найдется положительное число \(r_z\), такое, что из условий \(\rho(z, y) < r_z\), \(y \in Y\) следует, что \(y \in M\). Пусть \(G_z\) \(-\) множество всех \(x \in X\), таких, что \(2\rho(z, x) < r_z\); положим
\[
G = \bigcup\limits_{z \in M} G_z.
\]
Тогда \(G\) \(-\) открытое подмножество пространства \(X\) (по теоремам 2.1 и 2.8).

Поскольку \(z \in G_z\) при всех \(z \in M\), ясно, что \(M \subset G \cap Y\). Согласно нашему выбору окрестности \(G_z\), имеем \(G_z \cap Y \subset M\) при каждом \(z \in M\), так что \(G \cap Y \subset M\). Таким образом, \(M = G \cap Y\), и половина теоремы доказана.

Обратно, если множество \(G\) открыто в \(X\) и \(M = G \cap Y\), то каждая точка \(z \in M \subset G\) имеет окрестность \(G_z \subset G\). Тогда \(G_z \cap Y \subset G \cap Y = M\), так что множество \(M\) открыто относительно \(Y\).
\end{proof}

\begin{theorem}
Пусть \(Y \subset X\). Непустые подмножества \(M_1\), \(M_2\) подпространства \(Y\) открыты относительно \(Y\), \(M_1 \cap M_2 = \O\). Тогда существуют непересекающиеся открытые подмножества \(G_1\) и \(G_2\) пространства \(X\), такие, что \(M_1 = G_1 \cap Y\), \(M_2 = G_2 \cap Y\).
\end{theorem}

\begin{proof}
Используемые ниже обозначения взяты из доказательства теоремы 2.10.

Положим \(G_1 = \bigcup\limits_{z_1 \in M_1} G_{z_1}\), \(G_2 = \bigcup\limits_{z_2 \in M_2} G_{z_2}\). Тогда \(G_1\), \(G_2\) \(-\) искомые открытые подмножества пространства \(X\).

Покажем, что \(G_1 \cap G_2 = \O\). Для этого рассмотрим \(\rho(z_1, z_2)\), где \(z_1 \in M_1\), \(z_2 \in M_2\). Из того, что \(M_1 \cap M_2 = \O\), следует, что \(2\rho(z_1, z_2) \geq r_{z_1} + r_{z_2}\). В то же время, если некоторое множество \(G_{z_1}\) имеет общую точку \(\xi\) с некоторым \(G_{z_2}\), то
\[
2\rho(z_1, z_2) \leq 2\rho(z_1, \xi) + 2\rho(\xi, z_2) < r_{z_1} + r_{z_2}.
\]

Полученное противоречие завершает доказательство.
\end{proof}

\subsection{Открытые и замкнутые множества в \(\mathbb{R}^1\)}

Структура открытых и замкнутых множеств в том или ином метрическом пространстве может быть весьма сложной. Это относится даже к открытым и замкнутым множествам пространства \(\mathbb{R}^2\). Однако в одномерном случае исчерпывающее описание всех открытых множеств (а следовательно, и всех замкнутых) не представляет труда. Оно дается следующей теоремой.

\begin{theorem}
Всякое открытое множество на числовой прямой представляет собой объединение конечного или счетного числа попарно непересекающихся интервалов
\footnote[1]
{
Множества вида \((- \infty, \infty)\), \((a, \infty)\) и \((- \infty, b)\) мы при этом также включаем в число интервалов.
}.
\end{theorem}

\begin{proof}
Пусть \(G\) \(-\) открытое множество на прямой. Рассмотрим произвольную точку \(x \in G\). Точка \(x\) входит в множество \(G\) вместе с некоторым интервалом, содержащим точку \(x\). Мы построим сейчас наибольший интервал, содержащий точку \(x\) и содержащийся целиком в множестве \(G\).

Положим \(S = \{y \in G^c : y > x\}\). Если \(S = \O\), то \((x, \infty) \subset G\). Если \(S \neq \O\), то \(\exists b = \inf S\). Точка \(b \notin G\), так как у любой точки множества \(G\) есть окрестность, целиком входящая в \(G\) и не содержащая тем самым ни одной точки множества \(S\), а точка \(b\), как точная нижняя грань множества \(S\), в любой своей окрестности содержит точки из \(S\). В частности, \(x \neq b\). Очевидно также, что весь интервал \((x, b) \subset G\).

Аналогичное построение произведем слева от точки \(x\); мы получим там содержащийся в \(G\) интервал \((a, x)\), левый конец которого не входит в \(G\) (причем возможно, что \(a = - \infty\)).

Итак, по заданной точке \(x \in G\) мы построили интервал \((a, b)\), принадлежащий множеству \(G\) и такой, что его концы уже не входят в множество \(G\). Такого рода интервалы называются \textit{составляющими интервалами} открытого множества \(G\).

Если два составляющих интервала \((a_1, b_1)\) и \((a_2, b_2)\) имеют общую точку \(x\), то они целиком совпадают; действительно, неравенство, например \(b_1 < b_2\), невозможно, так как точка \(b_1\), с одной стороны, как внутренняя точка интервала \((x, b_2\)), должна принадлежать множеству \(G\), а с другой стороны, как концевая точка интервала \((x, b_1)\) она не может входить в \(G\). Поэтому все множество \(G\) есть объединение составляющих интервалов, не имеющих попарно общих точек.

Такое объединение не может быть более чем счетным; действительно, выбрав в каждом из этих интервалов произвольным образом рациональную точку, мы установим взаимно однозначное соответствие между этими интервалами и некоторым подмножеством \(\mathbb{Q}\). Теорема доказана.
\end{proof}

Так как замкнутые множества \(-\) это дополнения открытых, то из предыдущей теоремы следует, что всякое замкнутое множество на прямой получается выбрасыванием из прямой конечного или счетного числа интервалов.

\begin{theorem}
Пусть \(F\) \(-\) непустое замкнутое множество вещественных чисел, ограниченное сверху. Пусть \(y = \sup F\). Тогда \(y \in F\).
\end{theorem}

\begin{proof}
Допустим, что \(y \notin F\). Для каждого \(\varepsilon > 0\) существует точка \(x \in F\), такая, что \(y - \varepsilon \leq x \leq y\), так как иначе \(y - \varepsilon\) было бы верхней границей множества \(F\). Таким образом, каждая окрестность точки \(y\) содержит некоторую точку \(x \in F\), причем \(x \neq y\), так как \(y \notin F\). Следовательно, \(y\) \(-\) предельная точка множества \(F\), не принадлежащая \(F\), так что множество \(F\) не замкнуто. Но это противоречит условию теоремы.
\end{proof}

\section{Связные множества}

\begin{definition}
Множество \(M\) в метрическом пространстве \(X\) называется связным, если \textbf{не} существует двух открытых множеств \(A\) и \(B\) пространства \(X\), таких, что пересечение \( A \cap B\) пусто, пересечения \(A \cap M\) и \(B \cap M\) не пусты и \(M \subset A \cup B\).
\end{definition}

Чтобы сформулировать следующую теорему, мы будем говорить временно, что множество \(M\) связно относительно \(X\), если выполнены требования определения 3.1.

\begin{theorem}
Допустим, что \(M \subset Y \subset X\). Множество \(M\) связно относительно \(X\) в том и только в том случае, когда оно связно относительно \(Y\).
\end{theorem}

\begin{proof}
Если \(M\) не является связным относительно \(X\), то существуют множества \(A\) и \(B\), обладающие свойствами, указанными в определении, и рассмотрение множеств \(A \cap Y\) и \(B \cap Y\) показывает, что \(M\) не связно относительно \(Y\) (ср. с теоремой 2.10).

Обратное вытекает из теоремы 2.11.
\end{proof}

В силу этой теоремы мы сможем во многих ситуациях рассматривать связные множества как метрические пространства сами по себе, не обращая никакого внимания на объемлющее пространство.

\begin{definition}
Метрическое пространство \(X\) называется связным, если оно не является объединением двух непустых непересекающихся открытых множеств.
\end{definition}

\subsection{Связные множества в \(\mathbb{R}^1\)}

\begin{theorem}
Подмножество \(M\) пространства \(\mathbb{R}^1\) связно тогда и только тогда, когда \(M\) обладает следующим свойством: если \(x \in M\), \(y \in M\) и \(x < z < y\), то и \(z \in M\).
\end{theorem}

\begin{proof}
Допустим, что это условие не выполняется для некоторых чисел \(x\), \(y\), \(z\), т. е. \(x \in M\), \(y \in M\), \(x < z < y\), но \(z \notin M\). Если \(A\) есть множество всех \(\alpha < z\), a \(B\) \(-\) множество всех \(\beta > z\), то определение показывает, что \(M\) не связно.

Чтобы доказать обратное, допустим, что \(M\) не связно. Тогда существуют точки \(x \in M\), \(y \in M\), \(x < y\), и открытые непересекающиеся множества \(A\) и \(B\), такие, что \(x \in A\), \(y \in B\) и \(M \subset A \cup B\).

Положим \(S = A \cap [x, y]\). Множество \(S\) ограничено сверху, поскольку \(\forall s \in S \hspace{1mm} s \leq y\). Тогда \(\exists z = \sup S\), \(x \leq z \leq y\).

Если \(z \in B\) \(\Rightarrow\) \(\exists \varepsilon > 0 : (z - \varepsilon, z + \varepsilon) \subset B\) (ввиду того что \(B\) открыто) \(\Rightarrow\) \((z - \varepsilon, z + \varepsilon) \cap A = \O\) (ввиду того что \(A \cap B = \O\)) \(\Rightarrow\) \((z - \varepsilon, z + \varepsilon) \cap S = \O\) \(\Rightarrow\) \(z \neq \sup S\) \(-\) противоречие \(\Rightarrow\) \(z \notin B\). В частности, \(z \neq y\) \(\Leftrightarrow\) \(z < y\).

Если \(z \in A\) \(\Rightarrow\) \(\exists \varepsilon > 0 : (z - \varepsilon, z + \varepsilon) \subset A\) (ввиду того что \(A\) открыто) \(\Rightarrow\) \(\xi = z + {1 \over 2}\min(\varepsilon, y - z) \in S\). Но так как \(\xi > z\), то \(z \neq \sup S\) \(-\) противоречие \(\Rightarrow\) \(z \notin A\). В частности, \(z \neq x\) \(\Leftrightarrow\) \(x < z\).

Но так как \(M \subset A \cup B\), то \(z \notin M\), и доказательство закончено.
\end{proof}

\begin{consequence}
Множество \(M\) в \(\mathbb{R}^1\) связно тогда и только тогда, когда \(M\) \(-\) одно из следующих множеств (где \(a\) и \(b\) \(-\) вещественные числа, \(a \leq b\)):
\[
(-\infty, b), \hspace{1mm} (-\infty, b], \hspace{1mm} (a, \infty), \hspace{1mm} [a, \infty), \hspace{1mm} (-\infty, \infty), \hspace{1mm} (a, b), \hspace{1mm} [a, b), \hspace{1mm} (a, b], \hspace{1mm} [a, b].
\]
\end{consequence}

К какому из этих типов принадлежит множество \(M\), зависит от того, конечны или нет \(\inf M\) и \(\sup M\) и принадлежат ли они множеству \(M\).

Столь же простой характеристики связных множеств на плоскости, например, не существует.

\section{Плотные множества}

\begin{definition}
Множество \(M \subset X\) всюду плотно в \(X\), если каждая точка множества \(X\) является либо предельной точкой множества \(M\), либо принадлежит множеству \(M\) (либо и то, и другое).
\end{definition}

\begin{definition}
Метрическое пространство \(X\) называется сепарабельным, если оно содержит счетное всюду плотное подмножество \(M\).
\end{definition}

Пространство \(\mathbb{R}\) сепарабельно, так как множество рациональных чисел \(\mathbb{Q}\) всюду плотно в нем. \newline

Рассмотрим множество всех ограниченных последовательностей \(x = (x_1, ..., x_n, ...)\) вещественных чисел. Положив
\[
\rho(x, y) = \sup\limits_{k} |y_k - x_k|,
\]
мы получим метрическое пространство (справедливость аксиом очевидна).

\begin{theorem}
Пространство ограниченных последовательностей вещественных чисел несепарабельно.
\end{theorem}

\begin{proof}
Рассмотрим всевозможные последовательности, состоящие из нулей и единиц. Они образуют множество мощности континуума. Расстояние между двумя такими точками, определяемое формулой выше, равно \(1\).

Окружим каждую из этих точек открытым шаром радиуса \({1} \over {2}\). Эти шары не пересекаются. Если некоторое множество всюду плотно в пространстве ограниченных последовательностей, то каждый из построенных шаров должен содержать хотя бы по одной точке из этого множества, и, следовательно, оно не может быть счетным.
\end{proof}

\begin{theorem}
Любое подмножество \(M\) сепарабельного метрического пространства \(X\) само сепарабельно.
\end{theorem}

\begin{proof}
Пусть \(\{\xi_n\}_{n \in \mathbb{N}}\) \(-\) счетное всюду плотное множество в \(X\) и \(a_n = \inf\limits_{x \in M} {\rho(x, \xi_n)}\). Тогда для любых \(m, n \in \mathbb{N}\) найдется такая точка \(x_{nm} \in M\), что \(\rho(\xi_n, x_{nm}) < a_n + {1 \over m}\).

Пусть \(\varepsilon > 0\) и \(m\varepsilon > 3\). Для всякого \(x \in M\) найдется такое \(n \in \mathbb{N}\), что \(\rho(x, \xi_n) < {\varepsilon \over 3}\). В частности, \(a_n \leq {\varepsilon \over 3}\), а следовательно,
\[
\rho(x, x_{nm}) \leq \rho(x, \xi_n) + \rho(\xi_n, x_{nm}) < {\varepsilon \over 3} + a_n + {1 \over m} < {\varepsilon \over 3} + {\varepsilon \over 3} + {\varepsilon \over 3} = \varepsilon.
\]

Значит, не более чем счетное множество \(\{x_{nm}\}_{n, m \in \mathbb{N}}\) плотно в \(M\).
\end{proof}

\section{Изометричные пространства}

В теории множеств существенную роль играло понятие эквивалентности. Два эквивалентных множества, то есть два множества, находящихся во взаимно однозначном соответствии, с точки зрения чистой теории множеств были абсолютно равноправными, даже если они состояли из совершенно различных по природе элементов.

Но если два рассматриваемых нами множества являются метрическими пространствами (и интересуют нас именно как таковые), то теоретико-множественной эквивалентности уже недостаточно для того, чтобы мы считали такие два пространства равноправными, так как метрические соотношения у них могут быть совершенно различными.

Поэтому естественно ввести следующее определение:

\begin{definition}
Два метрических пространства называются изометричными, если между элементами этих пространств можно установить взаимно однозначное соответствие, сохраняющее расстояние между соответствующими парами элементов.
\end{definition}

Иными словами, если \((X, \rho)\) и \((X', \rho')\) \(-\) изометричные пространства и элементам \(x\), \(y\) множества \(X\) соответствуют элементы \(x'\), \(y'\) множества \(X'\), то \(\rho(x, y) = \rho'(x', y')\).

\(\)

В дальнейшем изометричные между собой пространства мы будем рассматривать просто как тождественные.

\end{document}\grid
