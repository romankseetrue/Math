\documentclass{article}

\usepackage[utf8]{inputenc}
\usepackage[russian]{babel}
\usepackage{amsthm}
\usepackage{amsmath} 
\usepackage{mathtools}
\usepackage{amssymb}
\usepackage[symbol]{footmisc}

\title{Cистема вещественных чисел \(\mathbb{R}\)}
\author{Роман Кушниренко}

\makeatletter
\newcommand*{\rom}[1]{\expandafter\@slowromancap\romannumeral #1@}
\makeatother

\renewcommand{\thefootnote}{\fnsymbol{footnote}}

\newtheorem{theorem}{Теорема}[section]
\newtheorem*{lemma}{Лемма}
\newtheorem{claim}{Утверждение}[section]
\newtheorem{example}{Пример}
\newtheorem{definition}{Определение}[section]
\newtheorem*{consequence}{Следствие}

\begin{document}
	\maketitle
	
	\section{Введение}

Система рациональных чисел \(\mathbb{Q}\) обладает многими недостатками. Например, не существует рационального числа \(p\), такого, что \(p^2=2\). Это делает необходимым введение так называемых «иррациональных чисел», которые часто записываются в виде бесконечных десятичных разложений, причем соответствующие конечные десятичные дроби считаются их приближениями. Так, последовательность	
\[
1; \hspace{1mm} 1,4; \hspace{1mm} 1,41; \hspace{1mm} 1,414; \hspace{1mm} 1,4142; \hspace{1mm} ...
\]
«стремится к \(\sqrt{2}\)». Но до тех пор, пока мы не определили 
иррациональное число \(\sqrt{2}\), остается открытым вопрос: к чему же все-таки стремится эта последовательность?

Главная цель этой главы и состоит в том, чтобы дать необходимое определение.\newline

Сначала покажем, что никакое \(p \in \mathbb{Q}\) не удовлетворяет уравнению
\[
p^2=2. \tag{1} \label{eq:eq1}
\]

Действительно, предположим, что это не так. Тогда \(\exists p = \frac{m}{n}\) (\(m,n \in \mathbb{Z}\); НОД\((m, n) = 1\)), удовлетворяющее уравнению \eqref{eq:eq1}.

Подставляя в уравнение, получаем
\[
m^2=2n^2.
\]
\[
m^2=2n^2 \Rightarrow 2 \mid m^2 \Rightarrow \mathbf{2 \mid m} \Rightarrow 4 \mid m^2 \Rightarrow 4 \mid 2n^2 \Rightarrow 2 \mid n^2 \Rightarrow \mathbf{2 \mid n}
\]

Таким образом, предположение о том, что выполнено равенство \eqref{eq:eq1}, заставляет нас заключить, что НОД\((m, n) \geq 2\),
вопреки нашему выбору \(m\) и \(n\). Значит, равенство \eqref{eq:eq1} невозможно при \(p \in \mathbb{Q}\).\newline

Исследуем теперь подробнее эту ситуацию.

\begin{claim}
\(A=\{p \in \mathbb{Q^{+}} : p^2 < 2\}\) не содержит наибольшего числа, т. е. \(\forall p \in A : \exists q \in A : q > p\).
\end{claim}

\begin{proof}
\(p \in A \Rightarrow p^2 < 2\). Выберем \(h \in \mathbb{Q^{+}}\), такое, что
\[
h < \min(1, \frac{2-p^2}{2p+1}).
\]
Положим \(q = p + h\). Тогда \(q \in \mathbb{Q^{+}}\), \(q > p\) и
\[
q^2 = p^2 + (2p + h)h < p^2 + (2p + 1)h < p^2 + (2 - p^2) = 2,
\]
так что \(q \in A\).
\end{proof}

\begin{claim}
\(B=\{p \in \mathbb{Q^{+}} : p^2 > 2\}\) не содержит наименьшего числа, т. е. \(\forall p \in B : \exists q \in B : q < p\).
\end{claim}

\begin{proof}
\(p \in B \Rightarrow p^2 > 2\). Положим
\[
q=p - \frac{p^2 - 2}{2p} = \frac{p}{2} + \frac{1}{p}.
\]
Тогда \(q \in \mathbb{Q^{+}}\), \(q < p\) и
\[
q^2 = p^2 - (p^2 - 2) + \frac{(p^2 - 2)^2}{4p^2} > p^2 - (p^2 - 2) = 2,
\]
так что \(q \in B\).
\end{proof}

Цель приведенного выше рассуждения \(-\) показать, что в системе рациональных чисел \(\mathbb{Q}\) имеются некоторые пробелы, несмотря на то, что 
\[
\forall p, q \in \mathbb{Q} : p < q \hspace{2mm} \exists r \in \mathbb{Q} : p < r < q \hspace{2mm} (r=\frac{p+q}{2}).
\]
Сейчас мы опишем предложенный Дедекиндом процесс\footnote[1]{Мы выбираем в дальнейшем теорию Дедекинда не потому, что она имеет какие-либо существенные преимущества перед другими, а лишь по той чисто внешней причине, что именно она принята в подавляющем большинстве наиболее распространённых руководств; таким образом, для читателя не представит затруднения отыскать такое пособие, где он сможет ознакомиться с деталями, пропущенными в нашем изложении.}, который позволяет заполнить эти пробелы и приводит нас к вещественным числам.

\section{Дедекиндовы сечения}

\begin{definition}
\(\alpha \subset \mathbb{Q}\) \(-\) сечение (точнее \(-\) сечение множества \(\mathbb{Q}\)), если
\renewcommand{\labelenumi}{\Roman{enumi}}
\begin{enumerate}
	\item \(\alpha \neq \varnothing\), \(\alpha \neq \mathbb{Q}\);
    \item \(q \in \mathbb{Q} \hspace{1mm} \land \hspace{1mm} q < p \in \alpha \Rightarrow q \in \alpha\);
    \item \(\forall p \in \alpha : \exists q \in \alpha : q > p\).
\end{enumerate}
\end{definition}

\break

\begin{theorem}
\(p \in \alpha \hspace{1mm} \land \hspace{1mm} q \notin \alpha \Rightarrow p < q\).
\end{theorem}

\begin{proof}
Предположим, что \(q \leq p\). Тогда \(p \in \alpha \xRightarrow{\text{\textit{\rom{2}}}} q \in \alpha\). Противоречие. 
\end{proof}

Принимая во внимание эту теорему:
\begin{itemize}
	\item \(p \in \alpha\) называют нижним числом сечения \(\alpha\);
	\item \(p \in \mathbb{Q} \hspace{1mm} \land \hspace{1mm} p \notin \alpha\) называют верхним числом сечения \(\alpha\).
\end{itemize}
 
Утверждение 1.2 показывает, что не всегда существует наименьшее верхнее число (рубеж). Однако для некоторых сечений рубеж действительно существует.

\begin{theorem}
\(r \in \mathbb{Q}\); \(\alpha = \{p \in \mathbb{Q} : p < r\}\). Тогда \(\alpha\) \(-\) сечение, а \(r\) \(-\) рубеж сечения \(\alpha\).
\end{theorem}

\begin{proof}
Ясно, что \(\alpha\) удовлетворяет условиям \text{\textit{\rom{1}}} и  \text{\textit{\rom{2}}} определения 2.1. Что касается \text{\textit{\rom{3}}}, то нужно лишь заметить, что для любого \(p \in \alpha\) мы имеем
\[
p < \frac{p+r}{2} < r,
\]
поэтому \(\frac{p+r}{2} \in \alpha\).

Далее, \(r \notin \alpha\). Поскольку неравенство \(p < r\) влечет за собой включение \(p \in \alpha\), то \(r\) \(-\) рубеж сечения \(\alpha\). 
\end{proof}

\begin{definition}
Сечение, построенное в теореме 2.2, называется рациональным сечением. Если мы захотим подчеркнуть, что \(\alpha\) есть рациональное сечение, связанное с числом \(r\) указанным образом, то будем писать \(\alpha = r^*\).
\end{definition}

Итак, все сечения множества \(\mathbb{Q}\) распадаются на два типа:
имеющие рубеж (рациональные) и не имеющие рубежа. При этом надо иметь в виду, что это распадение, очевидно, является внутренней структурной особенностью множества \(\mathbb{Q}\). Это есть факт, который остался бы в полной силе и в том случае, если бы мы и не помышляли никогда о введении каких-либо других чисел, кроме рациональных.

Этот факт прямо подсказывает нам дальнейший образ действия: каждому
сечению множества \(\mathbb{Q}\), не имеющему рубежа, следует поставить в соответствие некоторое новое, \textit{иррациональное число}, которое по определению будет его рубежом.

Таким образом, с помощью этого единого принципа мы сразу определяем всё множество иррациональных чисел. Вместе с существовавшими уже ранее рациональными числами они образуют множество всех вещественных чисел \(\mathbb{R}\) или континуум, который является теперь полностью определённым.

\section{Теория континуума}

Однако введённым нами принципом построения иррациональных чисел теория континуума, разумеется, не исчерпывается. Наоборот, здесь она в сущности только начинается. Программа работ, которые мы должны выполнить, прежде чем мы сможем говорить о действительно завершённой теории континуума, ещё чрезвычайно обширна. \newline

Мы должны:

\begin{itemize}
\item \textit{упорядочить} наш континуум, т. е. точно определить, при каких условиях данное вещественное число мы будем считать большим или меньшим другого данного вещественного числа;
\item определить действия над вещественными числами;
\item тщательным образом убедиться, что эти действия обладают всеми теми свойствами, к которым мы привыкли в области рациональных чисел;
\item найти способ убедиться, что определённый нами континуум действительно отвечает всем тем запросам практики и нашего наглядного представления, для удовлетворения которых он был нами построен.
\end{itemize}

Разумеется, детальное выполнение этой программы в рамках наших лекций совершенно невозможно. Да оно было бы и в высшей степени скучным. Мы в дальнейшем коснёмся лишь некоторых, принципиально наиболее важных моментов этой программы.

Прежде всего, введем отношение порядка в множестве сечений.

\begin{definition}
Пусть \(\alpha\), \(\beta\) \(-\) сечения. Мы будем писать \(\alpha = \beta\), если 
\[
	\begin{cases}
    	p \in \alpha \Rightarrow p \in \beta \\
      	q \in \beta \Rightarrow q \in \alpha
    \end{cases},      
\]
т. е. если эти два множества тождественно совпадают. В противном случае мы будем писать \(\alpha \neq \beta\).
\end{definition}

\begin{definition}
Пусть \(\alpha\) и \(\beta\) \(-\) сечения. Мы пишем \(\alpha < \beta\) (или \(\beta > \alpha\)), если \(\exists p \in \mathbb{Q} : p \in \beta \hspace{1mm} \land \hspace{1mm} p \notin \alpha\).
\end{definition}

\begin{theorem}
Пусть \(\alpha\), \(\beta\) \(-\) сечения. Тогда либо \(\alpha = \beta\), либо \(\alpha < \beta\), либо \(\beta < \alpha\).
\end{theorem}

\begin{proof}
Если \(\alpha\) и \(\beta\) \(-\) тождественно совпадают, то \(\alpha = \beta\). Если нет, то либо \(\alpha\) содержит \(p \in \mathbb{Q}\), не содержащееся в \(\beta\),
и в этом случае \(\beta < \alpha\), либо \(\beta\) содержит \(q \in \mathbb{Q}\), не содержащееся в \(\alpha\), и в этом случае \(\alpha < \beta\).

Возникает вопрос: возможно ли выполнение обоих неравенств? Ответ \(-\) нет. Чтобы это показать, предположим, что оба эти отношения имеют место.

\[
\alpha < \beta \Rightarrow \exists p \in \mathbb{Q} :
\begin{cases}
    p \in \beta \\
    p \notin \alpha
\end{cases}
\]

\[
\beta < \alpha \Rightarrow \exists q \in \mathbb{Q} :
\begin{cases}
    q \in \alpha \\
    q \notin \beta
\end{cases}
\]

\[
\begin{cases}
    p \in \beta \\
    q \notin \beta
\end{cases} \Rightarrow \hspace{3mm} p < q \hspace{2mm} (\text{теорема 2.1})
\]

\[
\begin{cases}
    q \in \alpha \\
    p \notin \alpha
\end{cases} \Rightarrow \hspace{3mm} q < p \hspace{2mm} (\text{теорема 2.1})
\]

Так как неравенства \(p<q\) и \(q<p\) не могут одновременно выполняться для рациональных чисел, мы пришли к противоречию.

\end{proof}

\begin{theorem}
Пусть \(\alpha\), \(\beta\), \(\gamma\) \(-\) сечения. Если \(\alpha < \beta\) и \(\beta < \gamma\), то \(\alpha < \gamma\).
\end{theorem}

Последние две теоремы показывают, что отношение \(<\) между сечениями действительно обладает теми свойствами, которые обычно связывают с понятием неравенства.

Теперь мы переходим к построению арифметики в множестве сечений.

\begin{theorem}
Пусть \(\alpha\), \(\beta\) \(-\) сечения. Тогда
\(\gamma = \{r \in \mathbb{Q} : r = p + q, p \in \alpha, q \in \beta\}\) \(-\) сечение.
\end{theorem}

\begin{proof}
Покажем, что \(\gamma\) удовлетворяет трем условиям определения 2.1.
\renewcommand{\labelenumi}{\Roman{enumi}}
\begin{enumerate}
	\item
\[
\begin{cases}
    \alpha \neq \varnothing \Rightarrow \exists p \in \alpha \\
    \beta \neq \varnothing \Rightarrow \exists q \in \beta
\end{cases}
\Rightarrow \hspace{3mm}
p + q \in \gamma \hspace{3mm}
\Rightarrow \hspace{3mm}
\gamma \neq \varnothing
\]
\[
\begin{cases}
    \alpha \neq \mathbb{Q} \Rightarrow \exists s \notin \alpha \\
    \beta \neq \mathbb{Q} \Rightarrow \exists t \notin \beta
\end{cases}
\Rightarrow \hspace{3mm}
s + t \notin \gamma \hspace{3mm}
\Rightarrow \hspace{3mm}
\gamma \neq \mathbb{Q}
\]
    \item Пусть \(r \in \gamma\), \(s < r\), \(s \in  \mathbb{Q}\). Тогда \(r = p + q\) при некоторых \(p \in \alpha\), \(q \in \beta\). Выберем \(t \in \mathbb{Q}\) так, что \(s = t + q\). Тогда \(t < p\), значит, \(t \in \alpha\), поэтому \(s \in \gamma\).
    \item Предположим, что \(r \in \gamma\). Тогда \(r = p + q\) при некоторых \(p \in \alpha\), \(q \in \beta\). Существует рациональное \(s > p\), такое, что \(s \in \alpha\). Значит, \(s + q \in \gamma\) и \(s + q > r\), так что \(r\) не является наибольшим рациональным числом в \(\gamma\).
\end{enumerate}
\end{proof}

\begin{definition}
Сечение \(\gamma\), построенное в предыдущей теореме, обозначается через \(\alpha + \beta\) и называется суммой \(\alpha\) и \(\beta\).
\end{definition}

\begin{theorem}
Пусть \(\alpha\), \(\beta\), \(\gamma\) \(-\) сечения. Тогда
\begin{enumerate}
    \item \(\alpha + \beta = \beta + \alpha\) (коммутативность);
    \item \((\alpha + \beta) + \gamma = \alpha + (\beta + \gamma)\) (ассоциативность);
    \item \(\alpha + 0^{*} = \alpha\) (нейтральный элемент).
\end{enumerate}
\end{theorem}

\begin{proof}
Для построения \(\alpha + \beta\) нужно взять множество всех рациональных чисел вида \(p + q\) (\(p \in \alpha\), \(q \in \beta\)). Для построения \(\beta +\alpha\) вместо \(p + q\) нужно брать \(q + p\). В силу закона коммутативности для сложения рациональных чисел, \(\alpha + \beta\) и \(\beta + \alpha\) \(-\) тождественные сечения, и свойство \textit{1} доказано.

Аналогичным образом закон ассоциативности для сложения рациональных чисел влечет за собой равенство \textit{2}.

Чтобы доказать \textit{3}, выберем \(r \in \alpha + 0^{*}\). Тогда \(r = p + q\) при некоторых \(p \in \alpha\), \(q \in 0^{*}\) (т. е. \(q < 0\)). Значит, \(p + q < p\), так что \(p + q \in \alpha\) и \(r \in \alpha\).

Теперь пусть \(r \in \alpha\). Выберем рациональное число \(s > r\), такое,
что \(s \in \alpha\). Положим \(q = r - s\). Тогда \(q < 0\), \(q \in 0^{*}\) и \(r = s + q\), так что \(r \in \alpha + 0^{*}\).

Таким образом, сечения \(\alpha + 0^{*}\) и \(\alpha\) совпадают.
\end{proof}

Мы здесь не будем останавливаться на определении других действий и доказательстве управляющих ими законов. Заметим только, что умножение лучше всего определять аналогично сложению, а вычитание и деление \(-\) как обратные действия.

Мы же теперь обратимся к последнему крайне важному в этом круге идей вопросу: как убедиться, что определённый нами континуум действительно являет собой ту непрерывность, сплошность, которая необходима ему как базе математического анализа и отсутствие которой у множества \(\mathbb{Q}\) заставило нас в своё время прибегнуть к введению иррациональных чисел?

Ответ содержится в следующей теореме.

\begin{theorem}[Дедекинд]

Пусть \(A\) и \(B\) \(-\) такие множества вещественных чисел, что
\begin{enumerate}
    \item Ни \(A\), ни \(B\) не пусты;
    \item Каждое вещественное число принадлежит или \(A\), или \(B\);
    \item Никакое вещественное число не принадлежит и \(A\), и \(B\);
    \item Если \(\alpha \in A\) и \(\beta \in B\), то \(\alpha < \beta\).
\end{enumerate}

Тогда существует одно (и только одно) вещественное число \(\gamma\), такое, что \(\alpha \leq \gamma\) при всех \(\alpha \in A\) и \(\gamma \leq \beta\) при всех \(\beta \in B\).

\end{theorem}

Именно существование \(\gamma\) (единственность тривиальна) составляет содержание этой важной теоремы. Оно показывает, что пробелы, которые мы обнаружили в системе рациональных чисел, теперь заполнены. Более того, если бы мы попытались повторить тот процесс, который привел нас от рациональных чисел к вещественным, и начали строить сечения (как в определении 2.1), элементами которых были бы вещественные числа, то каждое сечение имело бы наименьшее верхнее число, и мы смогли бы сразу же отождествить каждое сечение с наименьшим из его верхних чисел, не получив ничего нового.

По этой причине теорему Дедекинда иногда называют \textit{теоремой полноты} для вещественных чисел.

\begin{proof}[Существование]
Пусть \(\gamma\) \(-\) множество всех рациональных чисел \(p\), таких, что \(p \in \alpha\) при некотором \(\alpha \in A\). Мы должны проверить, что \(\gamma\) удовлетворяет условиям определения 2.1.

\renewcommand{\labelenumi}{\Roman{enumi}}
\begin{enumerate}
    \item Поскольку \(A\) непусто, \(\gamma \neq \varnothing\). Поскольку \(B\) непусто, \(\exists \beta \in B\). Если \(q \notin \beta\), то \(q \notin \alpha\) при любом \(\alpha \in A\) (ибо \(\alpha < \beta\)); значит, \(q \notin \gamma\), так что \(\gamma \neq \mathbb{Q}\).
    \item Если \(p \in \gamma\) и \(q < p\), то \(p \in \alpha\) при некотором \(\alpha \in A\); значит, \(q \in \alpha\), поэтому \(q \in \gamma\).
    \item Если \(p \in \gamma\), то \(p \in \alpha\) при некотором \(\alpha \in A\); значит, \(\exists q > p : q \in \alpha\); следовательно, \(q \in \gamma\).
\end{enumerate}

Таким образом, \(\gamma\) \(-\) вещественное число.

Ясно, что \(\alpha \leq \gamma\) при всех \(\alpha \in A\). Если бы при некотором \(\beta \in B\) оказалось, что \(\beta < \gamma\), то нашлось бы рациональное число \(p\), такое, что \(p \in \gamma\) и \(p \notin \beta\); но если \(p \in \gamma\), то \(p \in \alpha\) при некотором \(\alpha \in A\), а отсюда следует, что \(\beta < \alpha\), вопреки условию \textit{4}. Таким образом, \(\gamma \leq \beta\) при всех \(\beta \in B\), и доказательство существования закончено.

\end{proof}

\begin{proof}[Единственность]
Допустим, что имеются два числа \(\gamma_1\) и \(\gamma_2\), для которых выполнено заключение теоремы; пусть \(\gamma_1 < \gamma_2\).

Если \(\gamma_1 < \gamma_2\), то существует \(p \in \mathbb{Q}\), такое, что \(p \in \gamma_2\), \(p \notin \gamma_1\). Выберем \(r > p\) так, что \(r \in \gamma_2\). Поскольку \(r \in \gamma_2\) и \(r \notin r^{*}\), мы видим, что \(r^{*} < \gamma_2\). Поскольку \(p \in r^{*}\) и \(p \notin \gamma_1\), мы видим, что \(\gamma_1 < r^{*}\).

Таким образом, \(\exists r^{*} : \gamma_1 < r^{*} < \gamma_2\). Но тогда из неравенства \(r^{*} < \gamma_2\) следует, что \(r^{*} \in A\), в то время как неравенство \(\gamma_1 < r^{*}\) дает \(r^{*} \in B\). Это противоречит условию \textit{3}. Таким образом, существует не более чем одно число \(\gamma\) с требуемыми свойствами.

Заметим, что по ходу дела мы доказали, что \textit{для любых сечений \(\alpha\), \(\beta :\) \(\alpha < \beta\) существует рациональное сечение \(r^{*}\), такое, что \(\alpha < r^{*} < \beta\)}. Таким образом, между двумя любыми вещественными числами найдётся рациональное число, а значит, очевидно, и бесчисленное множество рациональных чисел. Это важное свойство множества \(\mathbb{Q}\) выражают обычно, говоря, что \(\mathbb{Q}\) \textit{всюду плотно} (на континууме).

\end{proof}

\begin{consequence}
В предположениях теоремы либо \(A\) содержит наибольшее число, либо \(B\) содержит наименьшее число.
\end{consequence}

\begin{proof}
Действительно, если \(\gamma \in A\), то \(\gamma\) \(-\) наибольшее число в \(A\); если \(\gamma \in B\), то \(\gamma\) \(-\) наименьшее в \(B\); в силу \textit{2}, одна из этих возможностей должна осуществиться, тогда как, в силу \textit{3}, они не могут осуществиться обе.
\end{proof}

\section{Основные леммы}

Логический фундамент математического анализа, таким образом, построен. Последовательно воздвигая на этой базе основы анализа, мы, разумеется, бываем вынуждены очень часто ссылаться на заложенный фундамент, то есть обращаться непосредственно к установленному нами определению вещественного числа. Это сопряжено с известными неудобствами, так как построение и исследование необходимых при этом сечений обычно бывает в достаточной мере громоздким.

Тот путь, которым наука находит выход из этого затруднения, в высшей степени поучителен, так как его можно считать типичным для всех подобного рода логических ситуаций, часто встречающихся в математической науке. В ходе развития математического анализа было замечено, что хотя непосредственно применять в рассуждениях определение вещественных чисел с помощью сечений приходится очень часто, однако многие из этих применений в формальном отношении весьма похожи друг на друга, так что фактически почти вся совокупность этих применений совершается по трём-четырём формальным схемам (наполняемым, конечно, всякий раз особым содержанием). Но если создалось такое положение, то было бы, конечно, очень неэкономно и весьма затруднило бы как построение, так и усвоение данной дисциплины, если бы мы стали десятки раз повторять одну и ту же логическую конструкцию, наполняя её только всякий раз новым предметным содержанием.

Математическая наука уже давно \(-\) и, конечно, с полным основанием \(-\) усвоила обычай во всех подобного рода случаях явно формулировать такую логическую схему в виде некоторого вспомогательного предложения (леммы), с тем, чтобы, раз навсегда доказав эту лемму, в дальнейшем уже иметь возможность не повторять всякий раз той формальной конструкции, которая лежит в основе этой леммы, а просто ссылаться на готовую лемму. Доказав три-четыре таких вспомогательных предложения, мы получаем возможность во всём дальнейшем уже почти никогда не возвращаться к конструкции сечений, всякий раз заменяя такую конструкцию ссылкой на одну из основных лемм, составляющих как бы небольшую группу мостов, соединяющих математический анализ с его логической базой. Само собой разумеется, что выбор этих основных лемм может быть различным в различных изложениях; однако во всех случаях можно рекомендовать не жалеть времени и сил на возможно тщательное усвоение возможно большего числа таких лемм, ибо каждая из них имеет своим назначением существенное облегчение работы в будущем, и поэтому затраченные на овладение ею усилия без всяких сомнений не пропадут даром.

Примерный тип формулировки и доказательства такого рода лемм мы покажем теперь на нескольких примерах.

\subsection{Лемма о существовании точной грани}

\begin{definition}
Пусть \(E \subset \mathbb{R}\). Если \(\exists y \in \mathbb{R}\), такой, что \(x \leq y\) при всех \(x \in E\), то мы будем говорить, что множество \(E\) ограничено сверху, а число \(y\) будем называть верхней границей множества \(E\).

Нижние границы определяются аналогичным образом.

Если множество \(E\) ограничено сверху и снизу, то оно называется ограниченным.

\end{definition}

\begin{definition}

Пусть \(E\) ограничено сверху. Предположим, что \(y \in \mathbb{R}\) обладает следующими свойствами:

\begin{enumerate}
    \item Является верхней границей множества \(E\);
    \item Если \(x < y\), то \(x\) не является верхней границей множества \(E\).
\end{enumerate}

Тогда \(y\) называется верхней гранью (точной верхней границей) множества \(E\). Как следует из 2, существует не более чем одно такое число \(y\). Мы будем употреблять сокращенное обозначение \(\sup\) для верхней грани.

Нижняя грань (\(\inf\)) любого множества \(E\), ограниченного снизу, определяется таким же образом.

\end{definition}

\begin{lemma}
Пусть \(E\) \(-\) непустое множество вещественных чисел, ограниченное сверху. Тогда \(\sup E\) существует.
\end{lemma}

\begin{proof}
Пусть \(A \subset \mathbb{R}\) \(-\) множество, определенное следующим образом: \(\alpha \in A\) в том и только в том случае, когда \(\exists x \in E\), такой, что \(\alpha < x\). Пусть \(B = \mathbb{R} \setminus A\).

Ясно, что никакой элемент множества \(A\) не является верхней границей множества \(E\), а каждый элемент множества \(B\) является верхней границей множества \(E\). Чтобы доказать существование верхней грани, достаточно поэтому доказать, что \(B\) содержит наименьшее число.

Проверим теперь, что \(A\) и \(B\) удовлетворяют предположениям теоремы Дедекинда.

Очевидно, что свойства \textit{2} и \textit{3} выполнены. Поскольку \(E\) непусто, \(\exists x \in E\) и каждое число \(\alpha < x\) принадлежит \(A\). Так как \(E\) ограничено сверху, \(\exists y : x \leq y\) \(\forall x \in E\), значит, \(y \in B\) и выполнено свойство \textit{1}. Если \(\alpha \in A\), то \(\exists x \in E\), такой, что \(\alpha < x\). Если \(\beta \in B\), то \(x \leq \beta\). Таким образом, \(\alpha < \beta\) при всех \(\alpha \in A\), \(\beta \in B\), и выполнено \textit{4}.

Итак, в силу следствия из теоремы Дедекинда, либо \(A\) содержит наибольшее число, либо \(B\) содержит наименьшее. Мы докажем, что первая возможность не может осуществиться.

Пусть \(\alpha\) \(-\) наибольшее число в \(A\). Тогда \(\exists x \in E : \alpha < x\). Выберем \(\alpha'\) так, что \(\alpha < \alpha' < x\). Поскольку \(\alpha' < x\), то \(\alpha' \in A\), так что \(\alpha\) не есть наибольшее число в \(A\).

Это завершает доказательство.
\end{proof}

\subsection{Лемма о монотонной последовательности}

\begin{definition}
Последовательность вещественных чисел \(\{s_n\}\) называется
\begin{itemize}
    \item монотонно возрастающей, если \(s_n \leq s_{n + 1}\) \(\forall n \in \mathbb{N}\);
    \item монотонно убывающей, если \(s_n \geq s_{n + 1}\) \(\forall n \in \mathbb{N}\).
\end{itemize}
\end{definition}

\begin{definition}
Множество всех точек \(s_n\) есть множество значений последовательности \(\{s_n\}\). Последовательность называется ограниченной, если множество ее значений ограничено.
\end{definition}

\begin{definition}
Последовательность вещественных чисел \(\{s_n\}\) называется сходящейся, если существует точка \(s \in \mathbb{R}\), обладающая следующим свойством: \(\forall \varepsilon > 0\) \(\exists N \in \mathbb{N}\), такое, что при \(n \geq N\) имеем \(| s_n - s| < \varepsilon\).

В этом случае мы будем говорить также, что \(\{s_n\}\) сходится к \(s\) или что \(s\) \(-\) предел последовательности \(\{s_n\}\), и будем писать \(s_n \to s\) или
\[
\lim_{n \to \infty} s_n = s.
\]

Если последовательность \(\{s_n\}\) не сходится, то говорят, что она расходится.
\end{definition}

\begin{lemma}
Всякая монотонная ограниченная последовательность вещественных чисел \(\{s_n\}\) имеет предел.
\end{lemma}

\begin{proof}
Допустим, что \(s_n \leq s_{n + 1}\) (в другом случае доказательство аналогично). Пусть \(E\) \(-\) множество значений последовательности \(\{s_n\}\). Если последовательность \(\{s_n\}\) ограничена, то пусть \(s\) \(-\) верхняя грань множества \(E\). Тогда \(s_n \leq s\) \(\forall n \in \mathbb{N}\).

Для любого \(\varepsilon > 0\) существует \(N \in \mathbb{N}\), такое, что \(s - \varepsilon < s_N \leq s\), так как иначе \(s - \varepsilon\) было бы верхней границей множества \(E\). Поскольку последовательность \(\{s_n\}\) возрастает, то при \(n \geq N\) имеем
\[
s - \varepsilon < s_n \leq s,
\]
откуда следует, что \(\{s_n\}\) сходится (к \(s\)).
\end{proof}

\subsection{Лемма о вложенных отрезках}

\begin{definition}
Отрезком \([a, b]\) называется совокупность всех \(x \in \mathbb{R}\), удовлетворяющих неравенствам \(a \leq x \leq b\).
\end{definition}

\begin{definition}
Последовательность отрезков
\[
[a_1, b_1], \hspace{1mm} [a_2, b_2], \hspace{1mm} ..., \hspace{1mm} [a_n, b_n], \hspace{1mm} ...
\]
называется стягивающейся, если она подчинена следующим двум требованиям:
\begin{enumerate}
    \item Каждый последующий отрезок целиком содержится в предыдущем:

    \[a_n \leq a_{n + 1} < b_{n + 1} \leq b_n \hspace{1mm} \forall n \in \mathbb{N}.\]

    \item Длины отрезков при безграничном возрастании их номеров стремятся к нулю:

    \[\lim_{n \to \infty} (b_n - a_n) = 0.\]

\end{enumerate}

\end{definition}

\begin{lemma}
Если последовательность отрезков \(-\) стягивающаяся, то \(\exists! \alpha \in \mathbb{R} : \alpha \in [a_n, b_n]\) \(\forall n \in \mathbb{N}\).
\end{lemma}

\begin{proof}[Существование]
В силу условия \textit{1} последовательность \(\{a_n\}\) монотонна и ограничена (последнее видно из того, что \(a_n < b_1\) \(\forall n \in \mathbb{N}\)). Тогда в силу леммы о монотонной последовательности она имеет предел; положим

\[\lim_{n \to \infty} a_n = \alpha.\]
Тогда
\[
\forall k, n \in \mathbb{N} \hspace{2mm}
a_n < b_k
\hspace{3mm} \Rightarrow \hspace{3mm}
\forall k \in \mathbb{N} \hspace{2mm}
\alpha \leq b_k
\hspace{3mm} \Rightarrow \hspace{3mm}
\forall n \in \mathbb{N} \hspace{2mm}
a_n \leq \alpha \leq b_n.
\]
Таким образом, число \(\alpha\) принадлежит каждому из отрезков \([a_n, b_n]\) и, следовательно, удовлетворяет условиям леммы.
\end{proof}

\begin{proof}[Единственность]
При наличии двух чисел \(\alpha\) и \(\beta\), удовлетворяющих условиям леммы, мы имели бы (предполагая для определённости \(\alpha < \beta\))
\[
a_n \leq \alpha < \beta \leq b_n \hspace{2mm} \forall n \in \mathbb{N},
\]
откуда
\[
b_n - a_n \geq \beta - \alpha > 0 \hspace{2mm} \forall n \in \mathbb{N},
\]
что противоречит свойству \textit{2} стягивающейся системы.
\end{proof}

\subsection{Лемма Гейне-Бореля}

\begin{definition}
Интервалом \((a, b)\) называется совокупность всех \(x \in \mathbb{R}\), удовлетворяющих неравенствам \(a < x < b\).
\end{definition}

\begin{definition}
(Вообще говоря, бесконечное) семейство интервалов \(M\) покрывает отрезок \([a, b]\), если каждая точка этого последнего лежит внутри по меньшей мере одного из интервалов системы \(M\).
\end{definition}

\begin{lemma}
Если система интервалов \(M\) покрывает отрезок \([a, b]\), то из неё можно выделить конечную подсистему \(M'\), также покрывающую отрезок \([a, b]\).
\end{lemma}

\begin{proof}
В самом деле, если отрезок \(\Delta_1 = [a, b]\) не допускает требуемого леммой конечного покрытия, то, разделив его пополам, мы можем утверждать, что по меньшей мере одна из двух половин также не допускает конечного покрытия (так как, очевидно, если бы обе его допускали, то допускал бы его и весь отрезок \(\Delta_1\)). Обозначим эту половину через \(\Delta_2\) (если обе половины не допускают конечного покрытия, то \(\Delta_2\) может означать любую из них) и разделим её опять пополам; снова мы можем утверждать, что по меньшей мере одна из двух половин (обозначим её через \(\Delta_3\)) не допускает конечного покрытия. Этот процесс мы можем продолжать безгранично, причём отрезки \(\{\Delta_n\}\), очевидно, образуют стягивающуюся систему отрезков.

На основании леммы о вложенных отрезках существует единственная точка \(\alpha\), принадлежащая всем этим отрезкам. Пусть \(\Delta\) \(-\) интервал системы \(M\), содержащий внутри себя точку \(\alpha\). Так как при \(n \to \infty\) длина отрезка \(\Delta_n\) стремится к нулю, причём \(\alpha \in \Delta_n\), то \(\Delta_n \subset \Delta\) для достаточно больших \(n\), и мы приходим к противоречию: отрезок \(\Delta_n\), по своему определению не допускающий конечного покрытия, покрывается одним интервалом \(\Delta\) системы \(M\). Этим противоречием и доказывается лемма Гейне-Бореля.
\end{proof}

\section{Основные выводы}

Теперь мы не только научились строить фундамент математического анализа, но, доказав четыре важнейших вспомогательных предложения, подготовили этот фундамент к наиболее удобному применению его в процессе дальнейшей конструкции основного здания. Какими методами воздвигается это здание, каковы основные понятия, идеи и способы рассуждения, употребительные при его построении, \(-\) всё это вы узнаете из следующих лекций.

\end{document}\grid