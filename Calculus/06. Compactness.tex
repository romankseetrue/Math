\documentclass{article}

\usepackage{cmap}
\usepackage[utf8]{inputenc}
\usepackage[russian, french]{babel}
\usepackage{amsthm}
\usepackage{amsmath}
\usepackage{mathtools}
\usepackage{amssymb}
\usepackage[symbol]{footmisc}
\usepackage{graphicx}
\usepackage{tikz}

\graphicspath{ {./images/} }

\title{Компактность}
\author{Роман Кушниренко}

\makeatletter
\newcommand*{\rom}[1]{\expandafter\@slowromancap\romannumeral #1@}
\makeatother

\renewcommand{\thefootnote}{\fnsymbol{footnote}}
\renewcommand*{\dotFFN}{}

\newtheorem{theorem}{Теорема}[section]
\newtheorem*{lemma}{Лемма}
\newtheorem{claim}{Утверждение}[section]
\newtheorem{example}{Пример}
\newtheorem{definition}{Определение}[section]
\newtheorem*{consequence}{Следствие}

\DeclareMathOperator*\lowlim{\underline{lim}}
\DeclareMathOperator*\uplim{\overline{lim}}

\begin{document}

	\selectlanguage{russian}

	\maketitle

	\section{Понятие компактности}

Фундаментальную роль в анализе играет следующий факт, известный под названием леммы Гейне-Бореля:

\textit{Если система интервалов \(M\) покрывает отрезок \([a, b]\), то из неё можно выделить конечную подсистему \(M'\), также покрывающую отрезок \([a, b]\).}

Можно показать, что это утверждение останется справедливым, если вместо интервалов рассматривать любые открытые множества.

Данная глава посвящена обобщению этого свойства отрезка числовой прямой.

\begin{definition}
Открытым покрытием множества \(M\) в метрическом пространстве \(X\) мы будем называть семейство \(\{G_\alpha\}\) открытых подмножеств пространства \(X\), такое, что \(M \subset \bigcup\limits_{\alpha}G_\alpha\).
\end{definition}

\begin{definition}
Подмножество \(K\) метрического пространства \(X\) называется компактным, если каждое открытое покрытие множества \(K\) содержит конечное подпокрытие\footnote[1]
{
Говоря точнее, требование состоит в том, что если \(\{G_\alpha\}\) \(-\) открытое покрытие множества \(K\), то имеется конечное число индексов \(\alpha_1, ..., \alpha_n\), таких, что
\[
K \subset G_{\alpha_1} \cup ... \cup G_{\alpha_n}.
\]
}. \newline
\end{definition}

Ранее мы заметили, что если \(M \subset Y \subset X\), то множество \(M\) может быть открытым относительно \(Y\), не будучи открытым относительно \(X\). Свойство множества \(M\) быть открытым зависит, таким образом, от пространства, в которое оно погружено. То же верно и в отношении свойства множества быть замкнутым.

Однако, как мы увидим, компактность \(-\) более удобное понятие. Чтобы сформулировать следующую теорему, мы будем говорить временно, что множество \(K\) компактно относительно \(X\), если выполнены требования определения 1.2.

\begin{theorem}
Допустим, что \(K \subset Y \subset X\). Множество \(K\) компактно относительно \(X\) в том и только в том случае, когда оно компактно относительно \(Y\).
\end{theorem}

\begin{proof}
Предположим, что множество \(K\) компактно относительно \(X\); пусть \(\{V_{\alpha}\}\) \(-\) семейство множеств, открытых относительно \(Y\), такое, что
\[
K \subset \bigcup\limits_{\alpha}V_\alpha.
\]
Ранее мы доказали, что при каждом \(\alpha\) существует множество \(G_\alpha\), открытое относительно \(X\), такое, что \(V_\alpha = Y \cap G_\alpha\); поскольку \(K\) компактно относительно \(X\), мы имеем
\[
K \subset G_{\alpha_1} \cup ... \cup G_{\alpha_n}  \tag{1} \label{eq:eq1}
\]
при некотором выборе конечного числа индексов \(\alpha_1, ..., \alpha_n\). Так как \(K \subset Y\), то из \eqref{eq:eq1} следует, что
\[
K \subset V_{\alpha_1} \cup ... \cup V_{\alpha_n}.  \tag{2} \label{eq:eq2}
\]
Тем самым доказано, что множество \(K\) компактно относительно \(Y\).

Обратно, допустим, что \(K\) компактно относительно \(Y\). Пусть \(\{G_{\alpha}\}\) \(-\) семейство открытых подмножеств пространства \(X\), покрывающее \(K\). Положим \(V_\alpha = Y \cap G_\alpha\). Тогда включение \eqref{eq:eq2} будет выполнено при некотором выборе \(\alpha_1, ..., \alpha_n\); так как \(V_\alpha \subset G_\alpha\), то \eqref{eq:eq1} следует из \eqref{eq:eq2}.

Доказательство закончено.
\end{proof}

В силу этой теоремы мы сможем во многих ситуациях рассматривать компактные множества как метрические пространства сами по себе, не обращая никакого внимания на объемлющее пространство. В частности, хотя почти бессмысленно говорить об \textit{открытых} пространствах или о \textit{замкнутых} пространствах (каждое метрическое пространство \(X\) служит открытым подмножеством самого себя и замкнутым подмножеством самого себя), имеет смысл говорить о \textit{компактных} метрических пространствах.

Доказательства следующих двух теорем тривиальны и остаются читателю в качестве упражнения.

\begin{theorem}
Для любого семейства компактных множеств \(\{K_{\alpha}\}\) множество \(\bigcap\limits_{\alpha}K_\alpha\) компактно.
\end{theorem}

\begin{theorem}
Для любого конечного семейства компактных множеств \(\{K_i\}_{i=1}^n\) множество \(\bigcup\limits_{i=1}^{n}K_i\) компактно.
\end{theorem}

\section{Свойства компактных пространств}

\begin{theorem}
Компактные подмножества метрических пространств замкнуты.
\end{theorem}

\begin{proof}
Пусть \(K\) \(-\) компактное подмножество метрического пространства \(X\). Мы докажем, что дополнение множества \(K\) есть открытое подмножество пространства \(X\).

Предположим, что \(x \in K^c\). Для каждой точки \(y \in K\) построим окрестность \(W_y\) радиуса, меньшего \({1 \over 2}\rho(x, y)\).

Ввиду того что \(K\) \(-\) компактно, найдется конечный набор точек \(y_1, ..., y_n\), принадлежащих множеству \(K\), таких, что
\[
K \subset W_{y_1} \cup ... \cup W_{y_n} = W.
\]

Положим
\[
\varepsilon = \min_{1 \leq m \leq n}{\rho(x, y_m)}.
\]

Ясно, что минимум конечного множества положительных чисел \(-\) положительное число, так что \(\varepsilon > 0\). Но тогда окрестность \(O_{\varepsilon \over 2}(x)\) не содержит общих точек с \(W\). Значит, \(O_{\varepsilon \over 2}(x) \subset K^c\), так что \(x\) \(-\) внутренняя точка множества \(K^c\). Теорема доказана.
\end{proof}

\begin{consequence}
Всякое компактное метрическое пространство \(X\) является полным.
\end{consequence}

\begin{proof}
Пусть \(X^*\) \(-\) пополнение компактного метрического пространства \(X\). По только что доказанной теореме пространство \(X\) как компактное подпространство является замкнутым в \(X^*\). Но замкнутое подпространство полного пространства полно. Следствие доказано.
\end{proof}

\begin{theorem}
Замкнутые подмножества компактных множеств компактны.
\end{theorem}

\begin{proof}
Допустим, что \(F \subset K \subset X\), множество \(F\) замкнуто (относительно \(X\)), а \(K\) \(-\) компактно. Пусть \(\{V_\alpha\}\) \(-\) открытое покрытие множества \(F\). Добавив к \(\{V_\alpha\}\) открытое множество \(F^c\), получим открытое покрытие множества \(K\). Ввиду компактности \(K\) некоторое конечное подсемейство этого покрытия также покрывает \(K\). Выбросим из этого подсемейства множество \(F^c\), если оно там есть; оставшиеся множества обязаны покрывать \(F\), что и требовалось.
\end{proof}

\begin{consequence}
Если \(F\) замкнуто, а \(K\) компактно, то \(F \cap K\) компактно.
\end{consequence}

\begin{theorem}
Если \(\{K_\alpha\}\) \(-\) семейство компактных подмножеств метрического пространства \(X\), такое, что пересечение любого конечного подсемейства семейства \(\{K_\alpha\}\) непусто, то и \(\bigcap\limits_{\alpha}K_\alpha\) непусто.
\end{theorem}

\begin{proof}
Зафиксируем множество \(K\) из семейства \(\{K_\alpha\}\) и положим \(G_\alpha = K^c_{\alpha}\). Предположим, что в \(K\) нет такой точки, которая принадлежала бы всем множествам \(K_\alpha\). Тогда множества \(G_\alpha\) образуют открытое покрытие множества \(K\). Так как \(K\) компактно, найдется конечный набор индексов \(\alpha_1, ..., \alpha_n\), такой, что
\[
K \subset G_{\alpha_1} \cup ... \cup G_{\alpha_n}.
\]
Но это означает, что множество
\[
K \cap K_{\alpha_1} \cap ... \cap K_{\alpha_n}
\]
пусто. Мы получили противоречие с условиями теоремы.
\end{proof}

\begin{consequence}
Если \(\{K_n\}\) \(-\) последовательность непустых компактных множеств, такая, что \(K_n \supset K_{n + 1}\) \(\forall n \in \mathbb{N}\), то и множество \(\bigcap\limits_{n=1}^{\infty}K_n\) непусто.
\end{consequence}

\begin{theorem}
Если \(M\) \(-\) бесконечное подмножество компактного множества \(K\), то \(M\) имеет предельную точку, принадлежащую \(K\).
\end{theorem}

\begin{proof}
Если бы никакая точка множества \(K\) не была предельной точкой множества \(M\), то каждая точка \(x \in K\) имела бы окрестность \(V_x\), содержащую не более одной точки множества \(M\) (а именно точку \(x\), если \(x \in M\)). Ясно, что никакое конечное подсемейство семейства \(\{V_x\}\) не может покрыть множество \(M\); то же верно и для \(K\), так как \(M \subset K\). Но это противоречит компактности множества \(K\).
\end{proof}

Для доказательства обратного утверждения нам понадобится следующая лемма.

\begin{lemma}
Если \(X\) \(-\) метрическое пространство со счетной базой, то из всякого его открытого покрытия можно выбрать не более чем счетное подпокрытие.
\end{lemma}

\begin{proof}
Пусть \(\{G_\alpha\}\) \(-\) некоторое открытое покрытие пространства \(X\). Тогда каждая точка \(x \in X\) содержится в некотором \(G_\alpha\). Пусть \(\{V_n\}\) \(-\) счетная база в \(X\). Для каждого \(x \in X\) существует такой элемент \(V_n(x)\) этой базы, что \(x \in V_n(x) \subset G_\alpha\). Совокупность выбранных таким образом множеств \(V_n(x)\) не более чем счетна и покрывает всё \(X\). Выбрав для каждого \(V_n(x)\) одно из содержащих его множеств \(G_\alpha\), мы получим не более чем счетное подпокрытие покрытия \(\{G_\alpha\}\). Теорема доказана.
\end{proof}

\begin{theorem}
Пусть каждое бесконечное подмножество метрического пространства \(X\) имеет предельную точку. Тогда \(X\) компактно.
\end{theorem}

\begin{proof}
Ранее мы доказали, что метрическое пространство \(X\), обладающее таким свойством, обязано быть сепарабельным, а значит оно имеет счетную базу. Следовательно, по лемме, каждое открытое покрытие пространства \(X\) содержит не более чем счетное подпокрытие \(\{G_n\}\), \(n \in \mathbb{N}\).

Предположим, что никакое конечное подсемейство семейства \(\{G_n\}\) не покрывает \(X\). Тогда множество \(F_n = (G_1 \cup ... \cup G_n)^c\) непусто при каждом \(n \in \mathbb{N}\). Более того, ясно, что все \(F_n\) замкнуты и образуют невозрастающую систему \(F_1 \supset F_2 \supset ...\). Покажем, что \(\bigcap\limits_{n=1}^{\infty}F_n \neq \varnothing\). Возможны два случая:

\begin{enumerate}
\item Начиная с некоторого номера \(N\)
\[
F_N = F_{N + 1} = ... .
\]
Тогда, очевидно,
\[
\bigcap\limits_{n=1}^{\infty}F_n = F_N \neq \varnothing.
\]
\item Среди \(F_n\) имеется бесконечно много попарно различных. При этом достаточно рассмотреть случай, когда все \(F_n\) различны между собой. Пусть
\[
x_n \in F_n \setminus F_{n + 1}.
\]
Последовательность \(\{x_n\}\) представляет собой бесконечное подмножество метрического пространства \(X\), следовательно, в силу условия, она должна иметь предельную точку, скажем, \(x\). Так как \(F_n\) содержит все точки \(x_n, x_{n + 1}, ...\), то \(x\) \(-\) предельная точка для \(F_n\) и в силу замкнутости \(F_n\), \(x \in F_n\). Следовательно,
\[
x \in \bigcap\limits_{n=1}^{\infty}F_n \neq \varnothing.
\]
\end{enumerate}

Итак, мы показали, что \(\bigcap\limits_{n=1}^{\infty}F_n \neq \varnothing\). Но тогда счетная система множеств \(\{G_n\}\) не покрывает всё пространство \(X\). Противоречие.
\end{proof}

\section{Вполне ограниченные множества}

\begin{definition}
Пусть \(\varepsilon\) \(-\) какое-нибудь положительное число, а \(M\) \(-\) множество, лежащее в метрическом пространстве \(X\). Множество \(M_{\varepsilon} \subset X\) называется \(\varepsilon\)-сетью для \(M\), если для любой точки \(x \in M\) найдется хотя бы одна точка \(y \in M_{\varepsilon}\), такая, что
\[
\rho(x, y) \leq \varepsilon\footnote[1]
{
Множество  \(M_{\varepsilon}\) не обязано содержаться в \(M\) и может даже не иметь с \(M\) ни одной общей точки, однако, имея для \(M\) некоторую \(\varepsilon\)-сеть \(M_{\varepsilon}\), можно построить \(2\varepsilon\)-сеть \(M'_{\varepsilon} \subset M\).
}.
\]
\end{definition}

\begin{theorem}
Если множество \(M \subset X\) имеет конечную \(\varepsilon\)-сеть при некотором данном \(\varepsilon > 0\), то оно ограничено.
\end{theorem}

\begin{proof}
В самом деле, пусть \(\{x_1, ..., x_n\}\) есть конечная \(\varepsilon\)-сеть множества \(M\). Тогда
\[
M = \bigcup\limits_{i=1}^{n}M_i,
\]
где \(M_i = B[x_i, \varepsilon] \cap M\). Итак, мы показали, что множество \(M\) представимо в виде объединения конечного числа ограниченных множеств. Следовательно, оно ограничено.
\end{proof}

\begin{definition}
Множество \(M \subset X\) называется вполне ограниченным, если для него при любом \(\varepsilon > 0\) существует конечная \(\varepsilon\)-сеть.
\end{definition}

\begin{theorem}
Всякое вполне ограниченное множество является ограниченным.
\end{theorem}

\begin{proof}
Очевидно.
\end{proof}

Обратное утверждение, вообще говоря, не верно. Однако, существуют такие метрические пространства, в которых эти понятия эквивалентны.

\begin{theorem}
Всякое ограниченное подмножество пространства \(\mathbb{R}^n\) является вполне ограниченным.
\end{theorem}

\begin{proof}
В самом деле, пусть подмножество пространства \(\mathbb{R}^n\) является ограниченным. Тогда его можно заключить в достаточно большой куб. Если такой куб разбить на кубики с ребром \(\varepsilon\), то вершины этих кубиков будут образовывать конечную \({\sqrt{n} \over 2}\varepsilon\)-сеть в исходном кубе, а значит, и подавно, в любом множестве, лежащем внутри этого куба.
\end{proof}

\begin{theorem}
Всякое вполне ограниченное множество является сепарабельным.
\end{theorem}

\begin{proof}
Действительно, построим для каждого \(n \in \mathbb{N}\) конечную \({1 \over n}\)-сеть. Тогда их объединение представляет собой счетное всюду плотное множество.
\end{proof}

\begin{theorem}
Если множество \(M \subset X\) компактно, то оно вполне ограничено.
\end{theorem}

\begin{proof}
Пусть \(\varepsilon > 0\). Совокупность окрестностей \(\{O_\varepsilon(x)\}\), построенных для каждой точки \(x \in M\), образует открытое покрытие множества \(M\), из которого, в силу компактности, можно извлечь конечное подпокрытие \(\{O_{\varepsilon}(x_1), ..., O_{\varepsilon}(x_n)\}\). Тогда множество \(\{x_1, ..., x_n\}\) \(-\) искомая конечная \(\varepsilon\)-сеть.
\end{proof}

\begin{consequence}
Всякое компактное множество ограничено.
\end{consequence}

\begin{consequence}
Всякое компактное множество сепарабельно.
\end{consequence}

\section{Признаки компактности}

Мы приведем теперь удобные условия, пригодные для проверки компактности конкретных метрических пространств.

\begin{theorem}[Хаусдорф]
Всякое вполне ограниченное и полное метрическое пространство \(X\) является компактным.
\end{theorem}

\begin{proof}
От противного. Пусть \(\{G_\alpha\}\) \(-\) открытое покрытие пространства \(X\), у которого нет конечного подпокрытия.

Построим вокруг каждой из точек, образующих \(1\)-сеть в \(X\), замкнутый шар радиуса \(1\). Так как эти шары покрывают все \(X\), а число их конечно, то по крайней мере один из них, назовем его \(B_1\), не покрывается никаким конечным подсемейством семейства \(\{G_\alpha\}\). Далее, выберем \(1 \over 2\)-сеть в \(B_1\) и вокруг каждой из точек этой сети построим замкнутый шар радиуса \(1 \over 2\). По крайней мере один из этих шаров, назовем его \(B_2\), не покрывается никаким конечным подсемейством семейства \(\{G_\alpha\}\). Далее, найдем замкнутый шар \(B_3\) с центром в \(B_2\) радиуса \(1 \over 4\), который не покрывается никаким конечным подсемейством семейства \(\{G_\alpha\}\) и т. д.

Рассмотрим теперь наряду с каждым шаром \(B_n\) замкнутый шар \(A_n\) с тем же центром, но в два раза большего радиуса. Легко видеть, что шары \(A_n\) вложены друг в друга. В силу полноты пространства \(X\) существует единственная точка, принадлежащая всем этим шарам:
\[
x = \bigcap\limits_{n=1}^{\infty}A_n.
\]

Пусть \(G\) \(-\) элемент покрытия \(\{G_\alpha\}\), содержащий внутри себя точку \(x\). Так как при \(n \to \infty\) радиус шара \(A_n\) стремится к нулю, причём \(x \in A_n\), то \(A_n \subset G\) для достаточно больших \(n\), и мы приходим к противоречию: шар \(A_n\), по своему определению не допускающий конечного покрытия, покрывается одним элементом \(G\) покрытия \(\{G_\alpha\}\). Этим противоречием и доказывается теорема.
\end{proof}

Следующая теорема дает нам исчерпывающее описание компактных множеств в \(\mathbb{R}^n\).

\begin{theorem}
Множество \(M \subset \mathbb{R}^n\) компактно тогда и только тогда, когда оно замкнуто и ограничено.
\end{theorem}

\begin{proof}
Очевидно.
\end{proof}

Закончим этот раздел обобщением известной леммы о предельной точке, которое очевидным образом следует из только что доказанной теоремы.

\begin{theorem}[Вейерштрасс]
Всякое ограниченное бесконечное подмножество пространства \(\mathbb{R}^n\) имеет предельную точку в \(\mathbb{R}^n\).
\end{theorem}

\end{document}\grid

