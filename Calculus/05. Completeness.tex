\documentclass{article}

\usepackage[utf8]{inputenc}
\usepackage[russian, french]{babel}
\usepackage{amsthm}
\usepackage{amsmath}
\usepackage{mathtools}
\usepackage{amssymb}
\usepackage[symbol]{footmisc}
\usepackage{graphicx}
\usepackage{tikz}

\graphicspath{ {./images/} }

\title{Полнота}
\author{Роман Кушниренко}

\makeatletter
\newcommand*{\rom}[1]{\expandafter\@slowromancap\romannumeral #1@}
\makeatother

\renewcommand{\thefootnote}{\fnsymbol{footnote}}
\renewcommand*{\dotFFN}{}

\newtheorem{theorem}{Теорема}[section]
\newtheorem*{lemma}{Лемма}
\newtheorem{claim}{Утверждение}[section]
\newtheorem{example}{Пример}
\newtheorem{definition}{Определение}[section]
\newtheorem*{consequence}{Следствие}

\DeclareMathOperator*\lowlim{\underline{lim}}
\DeclareMathOperator*\uplim{\overline{lim}}

\begin{document}

	\selectlanguage{russian}

	\maketitle

	\section{Фундаментальные последовательности}

Роль, которую играет в анализе вводимое ниже понятие фундаментальной последовательности и всё, с ним связанное, трудно переоценить.

\begin{definition}
Последовательность \(\{x_n\}\) точек метрического пространства \(X\) мы будем называть фундаментальной, если она удовлетворяет критерию Коши, то есть если для любого \(\varepsilon > 0\) существует такое число \(N \in \mathbb{N}\), что \(\rho(x_n, x_m) < \varepsilon\) для всех \(n \geq N\), \(m \geq N\).
\end{definition}

Заметим, что фундаментальность является внутренним свойством последовательности, зависящим только от её членов и ни от чего больше.

Изучим теперь простейшие, но часто используемые свойства фундаментальных последовательностей.

\begin{theorem}
Фундаментальная последовательность ограничена.
\end{theorem}

\begin{proof}
Если последовательность \(\{x_n\}\) фундаментальна, то найдётся номер \(N\) такой, что при \(m \geq N\), \(n \geq N\) будет выполняться неравенство \(\rho(x_m, x_n) < 1\). Положим номер \(m = N\). Тогда для любого \(n \geq N\) верно, что \(\rho(x_N, x_n) < 1\). Теперь положим
\[
r = \max \{1, \rho(x_N, x_1), \rho(x_N, x_2), ..., \rho(x_N, x_{N - 1})\}.
\]
Тогда \(x_n \in B[x_N, r]\) при \(n \in \mathbb{N}\).
\end{proof}

\begin{theorem}
Если фундаментальная последовательность имеет сходящуюся подпоследовательность, то она сходится.
\end{theorem}

\begin{proof}
Зафиксируем \(\varepsilon > 0\). Так как последовательность \(\{x_n\}\) фундаментальна, то найдётся номер \(N\) такой, что при \(m \geq N\), \(n \geq N\) выполняется условие \(2\rho(x_m, x_n) < \varepsilon\). Пусть \(\{x_{n_k}\}\) \(-\) сходящаяся подпоследовательность последовательности \(\{x_n\}\) и её предел равен \(x\). Тогда найдётся номер \(\xi\) такой, что \(n_\xi \geq N\) и \(2\rho(x_{n_\xi}, x) < \varepsilon\). Тогда для любого \(n \geq n_\xi\) имеем:
\[
\rho(x_n, x) \leq \rho(x_n, x_{n_\xi}) + \rho(x_{n_\xi}, x) < {\varepsilon \over 2} + {\varepsilon \over 2} = \varepsilon,
\]
а это означает, что последовательность \(\{x_n\}\) сходится к \(x\).
\end{proof}

\section{Полные метрические пространства}

В определении сходящейся последовательности есть один очень существенный изъян \(-\) его невозможно проверить непосредственно, не зная самого предела. В  связи  с  этим  возникает  вопрос:  \textit{можно  ли  установить  факт сходимости (или расходимости) последовательности,  не  привлекая ничего, кроме членов самой последовательности?} Исчерпывающий ответ на этот вопрос дается ниже.

Прежде всего заметим, что справедливо следующее утверждение.

\begin{theorem}
Всякая сходящаяся последовательность является фундаментальной.
\end{theorem}

\begin{proof}
Пусть \(\lim\limits_{n \to \infty} {x_n} = x\). Тогда для всякого \(\varepsilon > 0\) существует \(N \in \mathbb{N}\) такое, что \(2\rho(x_n, x) < \varepsilon\) для всех \(n \geq N\).

В силу неравенства треугольника
\[
\rho(x_m, x_n) \leq \rho(x_m, x) + \rho(x, x_n) < \varepsilon
\]
для всех \(m \geq N\) и \(n \geq N\).
\end{proof}

Обратное же утверждение верно не во всяком метрическом пространстве. Рассмотрим, например, открытый отрезок \((0, 1)\); он представляет собой метрическое пространство с обычной метрикой числовой оси. Последовательность
\(
\left\lbrace 1 \over n\right\rbrace
\),
очевидно, является фундаментальной в этом метрическом пространстве; но она не является в нем сходящейся.

Таким образом, фундаментальность слабее сходимости. Однако, существуют такие метрические пространства, в которых эти понятия эквивалентны.

\begin{definition}
Если в пространстве \(X\) всякая фундаментальная последовательность сходится, то это пространство называется полным.
\end{definition}

\begin{theorem}
\(\mathbb{R} ^ 1\) \(-\) полное метрическое пространство.
\end{theorem}

\begin{proof}
Если последовательность вещественных чисел \(\{x_n\}\) фундаментальна, то по теореме 1.1 она ограничена, поэтому по лемме Больцано-Вейерштрасса содержит сходящуюся подпоследовательность, следовательно, по теореме 1.2 сходится.
\end{proof}

\begin{theorem}
\(\mathbb{R} ^ n\) \(-\) полное метрическое пространство.
\end{theorem}

\begin{proof}
Полнота пространства \(\mathbb{R} ^ n\) непосредственно вытекает из полноты \(\mathbb{R} ^ 1\). В самом деле, пусть \(\{x_p\}\) \(-\) фундаментальная последовательность точек из \(\mathbb{R} ^ n\); это означает, что для каждого \(\varepsilon > 0\) найдется такое \(N \in \mathbb{N}\), что
\[
\sum\limits_{k = 1}^{n} {(x_{p}^{(k)} - x_{q}^{(k)})^2} < \varepsilon ^ 2
\]
при всех \(p, q \geq N\). Зафиксировав произвольное \(1 \leq k \leq n\), мы получаем
\[
{|x_{p}^{(k)} - x_{q}^{(k)}|} < \varepsilon
\]
при всех \(p, q \geq N\), то есть \(\{x_p^{(k)}\}\) \(-\) фундаментальная последовательность вещественных чисел. Положим
\[
x^{(k)} = \lim\limits_{p \to \infty} {x_p^{(k)}}.
\]
Тогда, очевидно,
\[
\lim\limits_{p \to \infty} {x_p} = x,
\]
где \(x = (x^{(1)}, ..., x^{(n)})\).
\end{proof}

Неограниченное количество дальнейших примеров дает следующая теорема.

\begin{theorem}
Пусть \((X, \rho)\) \(-\) полное метрическое пространство, a \(M \subset X\). Подпространство \((M, \rho)\) полно в том и только том случае, когда множество \(M\) замкнуто в \(X\).
\end{theorem}

\begin{proof}[Необходимость]
В самом деле, если бы множество \(M\) не было замкнуто в \(X\), то мы нашли бы последовательность \(\{x_n\}\) элементов множества \(M\), сходящуюся к некоторой точке \(\notin M\). Но всякая сходящаяся последовательность \(-\) фундаментальна, поэтому в силу полноты подпространства \((M, \rho)\) у последовательности \(\{x_n\}\) существовал бы предел \(\in M\). Таким образом, мы получили бы последовательность, имеющую два различных предела, \(-\) один в \(M\), другой вне \(M\), что невозможно.
\end{proof}

\begin{proof}[Достаточность]
Так как \((X, \rho)\) \(-\) полное метрическое пространство, то всякая фундаментальная последовательность \(\{x_n\}\) элементов множества \(M \subset X\) сходится, а ее предел принадлежит множеству \(M\) в силу предположенной замкнутости этого множества.
\end{proof}

\section{Теорема о вложенных шарах}

Как отличить полное метрическое пространство от неполного, кроме как непосредственно проверяя определение? Ответ на этот вопрос даёт следующая теорема.

\begin{theorem}
Для того чтобы метрическое пространство \(X\) было полным, необходимо и достаточно, чтобы в нем всякая последовательность вложенных друг в друга замкнутых шаров, радиусы которых стремятся к нулю, имела непустое пересечение.
\end{theorem}

\begin{proof}[Необходимость]
Пусть пространство \(X\) полно и пусть \(B[x_1, r_1]\) \(\supset\) \(B[x_2, r_2]\) \(\supset\) ... \(\supset\) \(B[x_n, r_n]\) \(\supset\) ... \(-\) последовательность вложенных друг в друга замкнутых шаров. Положим для краткости \(B_n = B[x_n, r_n]\).

Покажем, что последовательность \(\{x_n\}\) фундаментальна. В самом деле, если \(m > n\), то \(x_m \in B_n\), откуда \(\rho(x_n, x_m) \leq r_n\). Если теперь \(N \in \mathbb{N}\) таково, что \(r_k < \varepsilon\) при \(k \geq N\), то при \(m > n \geq N\) имеем \(\rho(x_n, x_m) \leq r_n < \varepsilon\), и фундаментальность последовательности установлена. Ввиду полноты пространства \(X\) отсюда вытекает, что существует \(x \in X : x_n \to x\). Так как \(x_m \in B_n\) при \(m > n\) и множество \(B_n\) замкнуто, получаем, что и
\[
x = \lim\limits_{m \to \infty} {x_m} = \lim\limits_{k \to \infty} {x_{n + k}}
\]
лежит в \(B_n\). Ввиду произвольности выбора \(n\) получаем, что \(x\) лежит в каждом \(B_n\).
\end{proof}

\begin{proof}[Достаточность]
Пусть \(\{x_n\}\) \(-\) фундаментальная последовательность. Докажем, что она имеет предел. В силу фундаментальности мы можем выбрать такую точку \(x_{n_1}\) нашей последовательности, что \(\rho(x_n, x_{x_{n_1}}) < {1 \over 2}\) при всех \(n \geq n_1\). Среди членов последовательности с номерами, большими \(n_1\), найдётся такая точка \(x_{n_2}\), что \(\rho(x_n, x_{x_{n_2}}) < {1 \over 4}\) при всех \(n \geq n_2\) и так далее. Продолжив описанный процесс неограничено, получим подпоследовательность \(\{x_{n_k}\}\) последовательности \(\{x_n\}\), обладающую свойством: \(\rho(x_n, x_{x_{n_k}}) < {1 \over {2 ^ k}}\) при всех \(n \geq n_k\).

При этом замкнутые шары \(B_k = B[x_{n_k}, {1 \over {2 ^ {k - 1}}}]\) вложены друг в друга. В самом деле, если \(x \in B_{k + 1}\), то \(\rho(x_{n_{k + 1}}, x) \leq {1 \over {2 ^ {k}}}\), поэтому
\[
\rho(x_{n_{k}}, x) \leq \rho(x_{n_{k}}, x_{n_{k + 1}}) + \rho(x_{n_{k + 1}}, x) \leq {1 \over {2 ^ {k - 1}}},
\]
следовательно, \(x \in B_k\).

Эта последовательность шаров имеет, по предположению, общую точку; обозначим ее \(x\). Ясно, что эта точка \(x\) служит пределом подпоследовательности \(\{x_{n_k}\}\). Но если фундаментальная последовательность содержит сходящуюся к \(x\) подпоследовательность, то она сама сходится к тому же пределу (теорема 1.2). Таким образом, \(x_n \to x\).
\end{proof}

\begin{claim}
Пересечение замкнутых вложенных шаров в предыдущей теореме состоит из одной точки.
\end{claim}

\begin{proof}
От противного. Пусть данное пересечение шаров содержит две различные точки, \(x\) и \(x'\). Тогда \(\rho(x, x') > 0\). Радиусы шаров стремятся к нулю, поэтому существует шар \(B[x_n, r_n]\) такой, что
\[
r_n < {\rho(x, x') \over 2}.
\]
Но такой шар может содержать лишь одну из точек, которые, по предположению, содержатся в данном пересечении шаров. Противоречие.
\end{proof}

Без условия о том, что радиусы шаров стремятся к нулю, теорема 3.1 неверна. В самом деле, пусть \(X = \mathbb{N}\), а
\[
\rho(n, m) =
\begin{cases}
0, & n = m; \\
1 + {1 \over \min(n, m)}, & n \neq m.
\end{cases}
\]
Нетрудно видеть, что все аксиомы метрического пространства выполнены; далее, поскольку расстояние между двумя различными точками всегда больше единицы, всякая фундаментальная последовательность обладает тем свойством, что начиная с какого-то места все ее члены совпадают, и такая последовательность очевидным образом сходится; значит, наше пространство полно. Имеем, наконец,
\[
B_n = B[n, 1 + {1 \over n}] = \{m \in \mathbb{N} : m \geq n\},
\]
и невзирая на то, что \(B_2\) \(\supset\) \(B_3\) \(\supset\) ..., пересечение всех шаров \(B_n\) пусто.

\end{document}\grid

