\documentclass{article}

\usepackage[utf8]{inputenc}
\usepackage[russian, french]{babel}
\usepackage{amsthm}
\usepackage{amsmath}
\usepackage{mathtools}
\usepackage{amssymb}
\usepackage[symbol]{footmisc}
\usepackage{graphicx}
\usepackage{tikz}

\graphicspath{ {./images/} }

\title{Элементы теории множеств}
\author{Роман Кушниренко}

\makeatletter
\newcommand*{\rom}[1]{\expandafter\@slowromancap\romannumeral #1@}
\makeatother

\renewcommand{\thefootnote}{\fnsymbol{footnote}}
\renewcommand*{\dotFFN}{}

\newtheorem{theorem}{Теорема}[section]
\newtheorem*{lemma}{Лемма}
\newtheorem{claim}{Утверждение}[section]
\newtheorem{example}{Пример}
\newtheorem{definition}{Определение}[section]
\newtheorem*{consequence}{Следствие}

\begin{document}

	\selectlanguage{russian}

	\maketitle

	\section{Множества и действия над ними}

В этой главе будет рассказано о том, что такое множества и какие действия можно выполнять над ними. К сожалению, основному понятию теории \(-\) понятию множества \(-\) нельзя дать строгого определения. Разумеется, можно сказать, что множество \(-\) это «совокупность», «собрание», «ансамбль», «коллекция», «семейство», «система», «класс» и т. д. Однако все это было бы не математическим определением, а скорее злоупотреблением словарным богатством русского языка. \newline

Возможны различные способы задания множества. Один из них состоит в том, что дается полный список элементов, входящих в множество. Но этот способ применим только к конечным множествам, да и то далеко не ко всем. Например, хотя множество всех рыб в океане и конечно, вряд ли его можно задать списком. А уж о составлении такого списка для бесконечного множества и думать нечего.

В тех случаях, когда множество нельзя задать при помощи списка, его задают путем указания некоторого характеристического свойства \(-\) такого свойства, что элементы множества им обладают, а все остальное на свете не обладает. \newline

В самых различных вопросах встречаются разбиения тех или иных множеств на попарно непересекающиеся подмножества (классы).

При разбиении множества на классы часто используют понятие \textit{эквивалентности} элементов. Для этого определяют, что значит «элемент \(x\) эквивалентен элементу \(y\)» (\(x \sim y\)), после чего объединяют эквивалентные элементы в один класс.

Однако не всякое понятие эквивалентности годится для такого разбиения. Например, назовем двух людей эквивалентными, если они знакомы друг с другом. Такое определение эквивалентности окажется неудачным. Ведь может случиться, что человек \(X\) знаком с человеком \(Y\), человек \(Y\) знаком с человеком \(Z\), а люди \(X\) и \(Z\) друг с другом незнакомы. Тогда нам придется сначала отнести в один класс людей \(X\) и \(Y\) (они друг с другом знакомы), потом в тот же класс включить и \(Z\) (он знаком с \(Y\)), и у нас в одном классе окажутся незнакомые друг с другом \(X\) и \(Z\).

Чтобы не было таких неприятностей, нужно, чтобы для понятия эквивалентности выполнялись следующие три условия:

\begin{enumerate}
  \item \(x \sim x\) для любого элемента \(x\) (рефлексивность).
  \item Если \(x \sim y\), то \(y \sim x\) (симметричность).
  \item Если \(x \sim y\) и \(y \sim z\), то \(x \sim z\) (транзитивность).
\end{enumerate}

\begin{theorem}
Пусть \(S\) \(-\) некоторое множество. Тогда выполнение этих условий необходимо и достаточно для того, чтобы \(S\) можно было разбить на классы.
\end{theorem}

\begin{proof}[Необходимость]
Всякое разбиение данного множества на классы определяет между элементами этого множества некоторое отношение эквивалентности. Действительно, если \(a \sim b\) означает «\(a\) находится в том же классе, что и \(b\)», то отношение \(\sim\) будет, как легко проверить, рефлексивным, симметричным и транзитивным.
\end{proof}

\begin{proof}[Достаточность]
Пусть \(\sim\) \(-\) некоторое отношение эквивалентности между элементами множества \(S\) и \(K_a = \{x \in S : x \sim a\}\). В силу свойства рефлексивности элемент \(a \in K_a\).

Покажем, что два класса \(K_a\) и \(K_b\) либо совпадают, либо не пересекаются. Пусть некоторый элемент \(c\) принадлежит одновременно и \(K_a\), и \(K_b\), т. е. \(c \sim a\) и \(c \sim b\). Тогда в силу симметричности \(a \sim c\) и в силу транзитивности \(a \sim b\).

Если теперь \(x\) \(-\) произвольный элемент из \(K_a\), т. е. \(x \sim a\), то в силу \(a \sim b\) и свойства транзитивности \(x \sim b\), т. е. \(x \in K_b\). Точно так же доказывается, что всякий элемент \(y \in K_b\) входит в \(K_a\). Таким образом, два класса \(K_a\) и \(K_b\), имеющих хотя бы один общий элемент, совпадают между собой.

Мы получили разбиение множества \(S\) на классы по заданному отношению эквивалентности.
\end{proof}

\section{Необыкновенная гостиница}

Домой я вернулся довольно поздно \(-\) вечер воспоминаний в клубе «Туманность Андромеды» затянулся далеко за полночь. Всю ночь меня мучили кошмары. То мне снилось, что меня проглотил огромный курдль, то грезилось, что я снова лечу на планету Дурдиотов и не знаю, как избежать тамошней страшной машины, превращающей людей в шестиугольники, то... В общем, никому не советую мешать старку с выдержанным медом. Неожиданный телефонный звонок вернул меня в мир реальности. Звонил старый друг и коллега по межзвездным странствиям профессор Тарантога.

«Срочное задание, дорогой Йон, \(-\) услышал я. \(-\) Астрономы обнаружили в космосе какой-то странный объект \(-\) от одной галактики до другой тянется таинственная черная линия. Никто не понимает, в чем дело. Самые лучшие радиотелескопы, нейтриноскопы и гравитоскопы не могут помочь в раскрытии тайны. Осталась надежда лишь на тебя. Срочно вылетай в направлении туманности АЦД-1587».

На другой день я получил из ремонта свою старую фотонную ракету, установил на нее ускоритель времени и электронного робота, знавшего все языки космоса и все рассказы о звездопроходцах (это гарантировало от скуки), и вылетел по заданию.

Когда робот исчерпал весь свой запас рассказов и начал повторяться (нет ничего хуже, чем электронный робот, десятый раз повторяющий старую историю), вдали показалась цель моего путешествия. Туманности, застилавшие таинственную линию, оказались позади, и предо мною предстала... гостиница «Космос».

Выяснилось, что межзвездные скитальцы выгонты, которым я когда-то соорудил небольшую планету, растащили и ее на мелкие части и вновь остались без пристанища. Тогда, чтобы больше не скитаться по чужим галактикам, они решили построить грандиозное сооружение \(-\) гостиницу для всех путешествующих по космосу. Эта гостиница протянулась через почти все галактики. Говорю «почти
все», потому что выгонты демонтировали некоторые необитаемые галактики, а из каждой оставшейся утащили по нескольку плохо лежавших созвездий.

Но гостиницу они отстроили на славу. В каждом номере были краны, из которых текла холодная и горячая плазма. При желании можно было на ночь распылиться, а утром портье собирал постояльцев по их атомным схемам.

А самое главное, \textit{в гостинице было бесконечно много номеров}. Выгонты надеялись, что теперь никому больше не придется слышать порядком надоевшую им за время скитаний фразу «свободных номеров нет».

Тем не менее мне не повезло. Когда я вошел в вестибюль гостиницы, первое, что бросилось в глаза, был плакат: «Делегаты съезда космозоологов регистрируются на 127-м этаже».

Так как космозоологи приехали из всех галактик, а их \(-\) бесконечное множество, то все номера оказались занятыми участниками съезда. Для меня места уже не хватило.

Администратор пытался, правда, поселить меня с кем-нибудь из космозоологов. Но когда я выяснил, что один предполагаемый сосед дышит фтором, а другой считает нормальной для себя температурой окружающей среды \(860 ^ \circ\), то вежливо отказался от столь «приятного» соседства.

К счастью, директором гостиницы был выгонт, хорошо помнивший услуги, которые я когда-то оказал этому племени. Он постарался устроить меня в гостинице, \(-\) ведь, ночуя в межзвездном пространстве, можно было схватить воспаление легких. После некоторых размышлений он обратился к администратору и сказал:

\(-\) Поселите его в №1.

\(-\) Куда же я дену жильца этого номера? \(-\) удивленно спросил администратор.

\(-\) А его переселите в №2. Жильца же из №2 отправьте в №3, из №3 \(-\) в №4 и т. д.

Тут только я оценил необыкновенные свойства гостиницы. Если бы в ней было лишь конечное число номеров, то жителю последнего номера пришлось бы перебраться в межзвездное пространство. А из-за того, что гостиница имела бесконечно много номеров, всем хватило места, и мне удалось вселиться, не лишив места никого из космозоологов.

Я не удивился, когда на другое утро мне предложили переселиться в №1000000. Просто в гостиницу прибыли запоздавшие космозоологи из галактики ВСК-3472, и надо было разместить еще 999999 жильцов. Но когда на третий день пребывания в гостинице я зашел к администратору заплатить за номер, у меня потемнело в глазах. К окошку тянулась очередь, конец которой терялся где-то около Магеллановых облаков. В очереди слышались голоса: «Меняю две марки туманности Андромеды на марку Сириуса!» «У кого есть марка Кита 57-го года космической эры?» В недоумении я обратился к администратору и спросил:

\(-\) А это кто такие?

\(-\) Межгалактический съезд филателистов.

\(-\) И много их?

\(-\) Бесконечное множество \(-\) по одному представителю от каждой галактики.

\(-\) Но как же их разместят, ведь космозоологи выедут только завтра?

\(-\) Не знаю, об этом сейчас будут говорить на пятиминутке у директора.

Однако задача оказалась весьма сложной, и пятиминутка (как это часто бывает и на Земле) затянулась на целый час. Наконец администратор вышел от директора и приступил к расселению. В первую очередь он приказал переселить жильца из №1 в №2. Мне это показалось странным, так как по имевшемуся опыту я знал, что такое переселение освобождало лишь один номер, а разместить надо было ни много ни мало, а бесконечное множество филателистов. Но администратор продолжал командовать:

\(-\) А жильца из №2 переселите в №4, из №3 \(-\) в №6, вообще из номера \(n\) \(-\) в номер \(2n\).

Теперь стал ясен его план: таким путем он освободил бесконечное множество нечетных номеров и мог расселять в них филателистов. В результате четные номера оказались занятыми космозоологами, а нечетные \(-\) филателистами (о себе не говорю \(-\) за три дня знакомства я так подружился с космозоологами, что был выбран почетным председателем их съезда; вместе со всеми космозоологами мне пришлось покинуть обжитый номер и переехать из №1000000 в №2000000). А мой знакомый филателист, стоявший в очереди 574-м, занял №1147. Вообще филателисты, стоявшие в очереди \(n\)-ми, занимали номер \(2n-1\).

На другой день положение с номерами стало легче \(-\) съезд космозоологов окончился, и они разъехались по домам. Я же переехал к директору гостиницы, в квартире которого освободилась одна комната. Но то, что хорошо для постояльцев, не всегда устраивает администрацию. Через несколько дней мой гостеприимный хозяин загрустил.

\(-\) В чем дело? \(-\) спросил я его.

\(-\) Половина номеров пустует. Финансовый план не выполняется.

Я, правда, не совсем понял, о каком финансовом плане шла речь, ведь плата поступала с бесконечного множества номеров, но тем не менее дал совет:

\(-\) А вы уплотните постояльцев, переселите их так, чтобы все номера оказались занятыми.

Это оказалось совсем просто сделать. Филателисты занимали лишь нечетные номера: 1, 3, 5, 7, 9 и т. д. Жильца из №1 оставили в покое. Из №3 переселили в №2, из №5 \(-\) в №3, из №7 \(-\) в №4 и т. д. В результате все номера вновь оказались заполненными, хотя ни один новый жилец не въехал.

Но неприятности директора на этом не кончились. Выяснилось, что выгонты не ограничились возведением гостиницы «Космос». Неугомонные строители соорудили еще бесконечное множество гостиниц, каждая из которых имела бесконечно много номеров. При этом они демонтировали так много галактик, что нарушилось межгалактическое равновесие, а это могло повлечь за собой весьма тяжкие последствия. Поэтому им было предложено закрыть все гостиницы, кроме нашей, и вернуть использованный материал на место. Но выполнение этого приказа было затруднено, поскольку все гостиницы (в том числе и наша) были заполнены. Предстояло переселить жильцов из бесконечного множества гостиниц, каждая из которых имела бесконечно много постояльцев, в одну гостиницу, да и та была уже заполнена.

\(-\) С меня хватит! \(-\) воскликнул директор, \(-\) Сначала я в полную гостиницу поместил одного постояльца, потом еще 999999, потом еще бесконечно много жильцов; а теперь от меня хотят, чтобы в нее вместилось еще бесконечное множество бесконечных множеств жильцов. Нет, гостиница не резиновая, пусть где хотят, там и помещают!

Но приказ есть приказ, и через пять дней надо было все подготовить к встрече новых постояльцев. Эти дни в гостинице никто не работал \(-\) все думали, как решить задачу. Был объявлен конкурс с премией \(-\) туристическим путешествием по одной из галактик. Но все предлагавшиеся решения отвергались, как неудачные. Так, младший повар предложил оставить жильца из первого номера нашей гостиницы в том же №1, из второго номера переселить в №1001, из третьего номера \(-\) в №2001 и т. д. После этого поселить жильцов второй гостиницы в №№2, 1002, 2002 и т. д. нашей гостиницы, жильцов третьей гостиницы \(-\) в №№3, 1003, 2003 и т. д. Проект был отвергнут, так как уже жители первых 1000 гостиниц займут все номера и некуда будет поселить жителей 1001-й гостиницы.

Мне вспомнилось по этому поводу, что, когда раболепные римские сенаторы предложили императору Тиберию переименовать в его честь месяц сентябрь в «тиберий» (предыдущие месяцы уже получили имена императоров Юлия и Августа), он язвительно спросил их: «А что же вы предложите тринадцатому цезарю?»

Неплохой вариант предложил бухгалтер гостиницы. Он посоветовал воспользоваться свойствами геометрической прогрессии и расселить постояльцев так: жителей первой гостиницы \(-\) в №№2, 4, 8, 16, 32 и т. д. (эти числа образуют геометрическую прогрессию со знаменателем 2). Жителей второй гостиницы \(-\) в №№3, 9, 27, 81 и т. д. (а это члены геометрической прогрессии со знаменателем 3). Так же предложил он расселять и жителей остальных гостиниц. Но директор спросил его:

\(-\) А для третьей гостиницы надо использовать прогрессию со знаменателем 4?

\(-\) Конечно, \(-\) ответил бухгалтер.

\(-\) Тогда ничего не получится, ведь в четвертом номере уже живет обитатель первой гостиницы, а теперь туда же надо вселить и жителя третьей гостиницы.

Настала моя очередь показать, что не зря в Звездной академии пять лет изучают математику.

\(-\) Воспользуйтесь простыми числами! Поселите жителей первой гостиницы в №№2, 4, 8, 16, ..., второй \(-\) в №№3, 9, 27, 81, ..., третьей \(-\) в №№5, 25, 125, 625, ..., четвертой \(-\) в №№7, 49, 343, ... .

\(-\) А не получится ли опять, что в один номер придется помещать двух постояльцев? \(-\) спросил директор.

\(-\) Нет! Ведь если взять два простых числа, то никакие их степени с натуральными показателями не могут оказаться равными. Если \(p\) и \(q\) \(-\) простые числа, причем \(p \neq q\), а \(m\) и \(n\) \(-\) натуральные числа, то \(p^m \neq q^n\).

Директор согласился со мной и тут же нашел усовершенствование предложенного способа, при котором использовались лишь два простых числа: 2 и 3. Именно, он предложил поселить жильца из \(m\)-го номера \(n\)-й гостиницы в номер \(2^m3^n\). Дело в том, что если \(m \neq p\) или \(n \neq q\), то \(2^m3^n \neq 2^p3^q\). Поэтому в один и тот же номер не поселятся двое.

Это предложение привело всех в восторг. Была решена задача, всем казавшаяся неразрешимой. Но премии не получили ни я, ни директор, \(-\) при наших решениях слишком много номеров оставались пустыми (у меня \(-\) такие номера, как 6, 10, 12 и вообще все номера, которые не были степенями простых чисел, а у директора \(-\) номера, которые нельзя записать в виде \(2^m3^n\)). Самое лучшее решение предложил один из филателистов \(-\) президент Математической академии галактики Лебедя.

Он посоветовал сначала составить таблицу, занумеровав ее строки номерами гостиниц, а столбцы \(-\) номерами комнат. Например, на пересечении четвертой строки и шестого столбца записывается шестая комната четвертой гостиницы. Вот эта таблица (вернее, ее левая верхняя часть, так как для записи всей таблицы надо бесконечно много строк и столбцов):

\begin{center}
\begin{tikzpicture}

\draw[xstep=1.25cm, ystep = 1cm, gray] (0, 0) grid (7.5, 6);

\node at (0.625, 5.5) {\((1; 1)\)};
\node at (1.875, 5.5) {\((1; 2)\)};
\node at (3.125, 5.5) {\((1; 3)\)};
\node at (4.375, 5.5) {...};
\node at (5.625, 5.5) {\((1; n)\)};
\node at (6.875, 5.5) {...};

\node at (0.625, 4.5) {\((2; 1)\)};
\node at (1.875, 4.5) {\((2; 2)\)};
\node at (3.125, 4.5) {\((2; 3)\)};
\node at (4.375, 4.5) {...};
\node at (5.625, 4.5) {\((2; n)\)};
\node at (6.875, 4.5) {...};

\node at (0.625, 3.5) {\((3; 1)\)};
\node at (1.875, 3.5) {\((3; 2)\)};
\node at (3.125, 3.5) {\((3; 3)\)};
\node at (4.375, 3.5) {...};
\node at (5.625, 3.5) {\((3; n)\)};
\node at (6.875, 3.5) {...};

\node at (0.625, 2.5) {...};
\node at (1.875, 2.5) {...};
\node at (3.125, 2.5) {...};
\node at (4.375, 2.5) {...};
\node at (5.625, 2.5) {...};
\node at (6.875, 2.5) {...};

\node at (0.625, 1.5) {\((m; 1)\)};
\node at (1.875, 1.5) {\((m; 2)\)};
\node at (3.125, 1.5) {\((m; 3)\)};
\node at (4.375, 1.5) {...};
\node at (5.625, 1.5) {\((m; n)\)};
\node at (6.875, 1.5) {...};

\node at (0.625, 0.5) {...};
\node at (1.875, 0.5) {...};
\node at (3.125, 0.5) {...};
\node at (4.375, 0.5) {...};
\node at (5.625, 0.5) {...};
\node at (6.875, 0.5) {...};

\end{tikzpicture}
\end{center}

\(-\) А теперь расселяйте обитателей по квадратам, \(-\) сказал математик-филателист.

\(-\) Как? \(-\) не понял директор.

\(-\) По квадратам! В №1 поселяется жилец из (1; 1), то есть из первого номера первой гостиницы; в №2 \(-\) из (1; 2), то есть из второго номера первой гостиницы; в №3 \(-\) из (2; 2) \(-\) второго номера второй гостиницы и в №4 \(-\) из (2; 1) \(-\) первого номера второй гостиницы. Тем самым будут расселены жильцы из верхнего левого квадрата со стороной 2. После этого в №5 поселяем жильца из (1; 3), в №6 \(-\) из (2; 3), в №7 \(-\) из (3; 3), в №8 \(-\) из (3; 2), в №9 \(-\) из (3; 1). (Эти номера образуют квадрат со стороной 3.)

И, взяв листок бумаги, он набросал на нем следующую схему расселения:

\begin{center}
\begin{tikzpicture}

\draw[xstep=1.25cm, ystep = 1cm, gray] (0, 0) grid (7.5, 6);

\node at (0.625, 5.5) {\((1; 1)\)};
\node at (1.875, 5.5) {\((1; 2)\)};
\node at (3.125, 5.5) {\((1; 3)\)};
\node at (4.375, 5.5) {...};
\node at (5.625, 5.5) {\((1; n)\)};
\node at (6.875, 5.5) {...};

\node at (0.625, 4.5) {\((2; 1)\)};
\node at (1.875, 4.5) {\((2; 2)\)};
\node at (3.125, 4.5) {\((2; 3)\)};
\node at (4.375, 4.5) {...};
\node at (5.625, 4.5) {\((2; n)\)};
\node at (6.875, 4.5) {...};

\node at (0.625, 3.5) {\((3; 1)\)};
\node at (1.875, 3.5) {\((3; 2)\)};
\node at (3.125, 3.5) {\((3; 3)\)};
\node at (4.375, 3.5) {...};
\node at (5.625, 3.5) {\((3; n)\)};
\node at (6.875, 3.5) {...};

\node at (0.625, 2.5) {...};
\node at (1.875, 2.5) {...};
\node at (3.125, 2.5) {...};
\node at (4.375, 2.5) {...};
\node at (5.625, 2.5) {...};
\node at (6.875, 2.5) {...};

\node at (0.625, 1.5) {\((n; 1)\)};
\node at (1.875, 1.5) {\((n; 2)\)};
\node at (3.125, 1.5) {\((n; 3)\)};
\node at (4.375, 1.5) {...};
\node at (5.625, 1.5) {\((n; n)\)};
\node at (6.875, 1.5) {...};

\node at (0.625, 0.5) {...};
\node at (1.875, 0.5) {...};
\node at (3.125, 0.5) {...};
\node at (4.375, 0.5) {...};
\node at (5.625, 0.5) {...};
\node at (6.875, 0.5) {...};

\draw[red, ->, thick] (1.875, 5.125) -- (1.875, 4.875);
\draw[red, ->, thick] (3.125, 5.125) -- (3.125, 4.875);
\draw[red, ->, thick] (5.625, 5.125) -- (5.625, 4.875);

\draw[red, ->, thick] (3.125, 4.125) -- (3.125, 3.875);
\draw[red, ->, thick] (5.625, 4.125) -- (5.625, 3.875);

\draw[red, ->, thick] (5.625, 3.125) -- (5.625, 2.875);

\draw[red, ->, thick] (5.625, 2.125) -- (5.625, 1.875);

\draw[red, ->, thick] (1.375, 4.5) -- (1.125, 4.5);

\draw[red, ->, thick] (1.375, 3.5) -- (1.125, 3.5);
\draw[red, ->, thick] (2.625, 3.5) -- (2.375, 3.5);

\draw[red, ->, thick] (1.375, 1.5) -- (1.125, 1.5);
\draw[red, ->, thick] (2.625, 1.5) -- (2.375, 1.5);
\draw[red, ->, thick] (3.875, 1.5) -- (3.625, 1.5);
\draw[red, ->, thick] (5.125, 1.5) -- (4.875, 1.5);

\end{tikzpicture}
\end{center}

\(-\) Неужели для всех хватит места? \(-\) усомнился директор.

\(-\) Конечно. Ведь в первые \(n^2\) номеров мы поселяем при этой схеме жильцов из первых \(n\) номеров первых \(n\) гостиниц. Поэтому рано или поздно каждый жилец получит номер. Например, если это жилец из №136 гостиницы №217, то он получит номер на 217-м шаге. Легко даже сосчитать этот номер. Он равен \(217^2 - 136 + 1\). Вообще, если жилец занимает номер \(n\) в \(m\)-й гостинице, то при \(n \geq m\) он займет номер \((n - 1)^2 + m\), а при \(n < m\) \(-\) номер \(m^2 - n + 1\).

Предложенный проект и был признан наилучшим \(-\) все жители из всех гостиниц были поселены в нашей гостинице, и ни один ее номер не пустовал. Математику-филателисту досталась премия \(-\) туристическая путевка в галактику ЛЦР-287.

В честь столь удачного размещения директор гостиницы устроил прием, на который пригласил всех ее жильцов. Этот прием также не обошелся без осложнений. Обитатели комнат с четными номерами задержались на полчаса, и, когда они появились, оказалось, что все стулья заняты, хотя гостеприимный хозяин поставил по стулу на каждого гостя. Пришлось подождать, пока все пересели на новые места и освободили необходимое количество стульев (разумеется, ни одного нового стула в зал не внесли). Зато когда стали подавать мороженое, то каждый гость получил по две порции, хотя повар заготовил в точности по одной порции на гостя. Надеюсь, что теперь читатель сам поймет, как все это случилось.

После конца приема я сел в свою фотонную ракету и полетел на Землю. Мне нужно было рассказать всем земным космонавтам о новом пристанище в космосе. Кроме того, я хотел проконсультироваться с виднейшими математиками Земли и моим другом профессором Тарантогой о свойствах бесконечных множеств.

\section{Мощность множеств}

Для конечных множеств задача сравнения решается просто. Чтобы узнать, одинаково ли число элементов в двух множествах, достаточно пересчитать их. Если получатся одинаковые числа, то, значит, в обоих множествах поровну элементов. Но для бесконечных множеств такой способ не годится, ибо, начав пересчитывать элементы бесконечного множества, мы рискуем посвятить этому делу всю свою жизнь и все же не закончить начатого предприятия.

Но и для конечных множеств метод пересчета не всегда удобен. Пойдем, например, на танцплощадку. Как узнать, поровну ли здесь юношей и девушек? Конечно, можно попросить юношей отойти в одну сторону, а девушек в другую, и заняться подсчетом как тех, так и других. Но, во-первых, мы получим при этом избыточную информацию, нас не интересует, сколько здесь юношей и девушек, а интересует лишь, поровну ли их. Во-вторых, не для того собралась молодежь на танцплощадке, чтобы стоять и ждать конца пересчета, а для того, чтобы потанцевать.

Ну что же. Удовлетворим их желание и попросим оркестр сыграть какой-нибудь танец, который все умеют танцевать. Тогда юноши пригласят девушек к танцу и... наша задача будет решена. Ведь если окажется, что все юноши и все девушки танцуют, то есть если вся молодежь разбилась на танцующие пары, то ясно, что на площадке ровно столько же юношей, сколько и девушек. \newline

Существование взаимно однозначного соответствия для конечных множеств равносильно тому, что у них поровну элементов. Важнейшим поворотным пунктом в теории множеств был момент, когда Кантор решил применить идею взаимно однозначного соответствия для сравнения бесконечных множеств.

Иными словами, по Кантору два (быть может и бесконечных) множества \(A\) и \(B\) имеют поровну элементов, если между этими множествами можно установить взаимно однозначное соответствие.

Обычно математики не говорят, что «множества \(A\) и \(B\) имеют поровну элементов», а говорят, что «\(A\) и \(B\) имеют одинаковую \textit{мощность}» или «множества \(A\) и \(B\) \textit{эквивалентны}» и обозначают как \(A \sim B\).

Таким образом, для бесконечных множеств слово «мощность» значит то же самое, что для конечных множеств «число элементов».

Еще до Кантора к понятию взаимно однозначного соответствия пришел чешский ученый Больцано. Но он отступил перед трудностями, к которым вело это понятие. Как мы вскоре увидим, после принятия принципа сравнения бесконечных множеств с помощью взаимно однозначного соответствия пришлось расстаться со многими догмами. \newline

Основной догмой, которую пришлось отбросить, было положение, установленное на самой заре развития математики: \textit{«часть меньше целого»}. Это положение безусловно верно для конечных множеств, но для бесконечных множеств оно уже теряет силу.

Вспомните, как расселил директор необыкновенной гостиницы космозоологов по четным номерам. При этом расселении жилец из номера \(n\) переезжал в номер \(2n\). Но эта схема устанавливает взаимно однозначное соответствие между множеством натуральных чисел и его частью \(-\) множеством четных чисел. А мы договорились считать, что множества, между которыми можно установить взаимно однозначное соответствие, содержат поровну элементов.

Значит, множество натуральных чисел содержит столько же элементов, сколько и его часть \(-\) множество четных чисел. \newline

Все множества, которые имеют столько же элементов, сколько имеет множество натуральных чисел, называют \textit{счетными}. \newline

Иногда для того, чтобы установить счетность того или иного множества, надо проявить изобретательность. Возьмем, например, множество всех целых чисел. Если мы попробуем нумеровать его по порядку, начиная с какого-нибудь места, то никогда эту нумерацию не закончим. Поэтому все числа до выбранного места останутся незанумерованными. Чтобы не пропустить при нумерации ни одного числа, надо записать это множество в виде двух строк:

\begin{center}
\begin{tikzpicture}

\draw[xstep=1.25cm, ystep = 1cm, gray] (0, 0) grid (7.5, 2);

\node at (0.625, 1.5) {\(0\)};
\node at (1.875, 1.5) {\(1\)};
\node at (3.125, 1.5) {\(2\)};
\node at (4.375, 1.5) {\(3\)};
\node at (5.625, 1.5) {\(4\)};
\node at (6.875, 1.5) {...};

\node at (0.625, 0.5) {\(-1\)};
\node at (1.875, 0.5) {\(-2\)};
\node at (3.125, 0.5) {\(-3\)};
\node at (4.375, 0.5) {\(-4\)};
\node at (5.625, 0.5) {\(-5\)};
\node at (6.875, 0.5) {...};

\draw[red, ->, thick] (0.625, 1.125) -- (0.625, 0.875);
\draw[red, ->, thick] (1.875, 1.125) -- (1.875, 0.875);
\draw[red, ->, thick] (3.125, 1.125) -- (3.125, 0.875);
\draw[red, ->, thick] (4.375, 1.125) -- (4.375, 0.875);
\draw[red, ->, thick] (5.625, 1.125) -- (5.625, 0.875);

\draw[red, ->, thick] (1.125, 0.875) -- (1.375, 1.125);
\draw[red, ->, thick] (2.375, 0.875) -- (2.625, 1.125);
\draw[red, ->, thick] (3.625, 0.875) -- (3.875, 1.125);
\draw[red, ->, thick] (4.875, 0.875) -- (5.125, 1.125);
\draw[red, ->, thick] (6.125, 0.875) -- (6.375, 1.125);

\end{tikzpicture}
\end{center}
и нумеровать по столбцам. При этом 0 получит №1, \(-\)1 \(-\) №2, 1 \(-\) №3, \(-\)2 \(-\) №4 и т. д. Иными словами, все положительные числа и нуль нумеруются нечетными числами, а все отрицательные целые числа \(-\) четными. \newline

Занумеровать рациональные числа в порядке возрастания их величины невозможно. Однако если отказаться от расположения рациональных чисел в порядке возрастания, то занумеровать их все же удается. Сделаем так: выпишем сначала все положительные дроби со знаменателем 1, потом все положительные дроби со знаменателем 2, потом со знаменателем 3 и т. д. У нас получится таблица следующего вида:

\begin{center}
\begin{tikzpicture}

\draw[xstep=1.25cm, ystep = 1cm, gray] (0, 0) grid (5, 4);

\node at (0.625, 3.5) {\(1 \over 1\)};
\node at (1.875, 3.5) {\(2 \over 1\)};
\node at (3.125, 3.5) {\(3 \over 1\)};
\node at (4.375, 3.5) {...};

\node at (0.625, 2.5) {\(1 \over 2\)};
\node at (1.875, 2.5) {\(2 \over 2\)};
\node at (3.125, 2.5) {\(3 \over 2\)};
\node at (4.375, 2.5) {...};

\node at (0.625, 1.5) {\(1 \over 3\)};
\node at (1.875, 1.5) {\(2 \over 3\)};
\node at (3.125, 1.5) {\(3 \over 3\)};
\node at (4.375, 1.5) {...};

\node at (0.625, 0.5) {...};
\node at (1.875, 0.5) {...};
\node at (3.125, 0.5) {...};
\node at (4.375, 0.5) {...};

\draw[red, ->, thick] (1.875, 3.125) -- (1.875, 2.875);
\draw[red, ->, thick] (3.125, 3.125) -- (3.125, 2.875);

\draw[red, ->, thick] (3.125, 2.125) -- (3.125, 1.875);

\draw[red, ->, thick] (1.375, 2.5) -- (1.125, 2.5);

\draw[red, ->, thick] (1.375, 1.5) -- (1.125, 1.5);
\draw[red, ->, thick] (2.625, 1.5) -- (2.375, 1.5);

\end{tikzpicture}
\end{center}

Ясно, что в этой таблице мы встретим любое положительное рациональное число, и притом не один раз. Теперь приступим к нумерации. Для этого вспомним последний подвиг директора необыкновенной гостиницы, который расселил в ней жителей из бесконечного множества таких же гостиниц. Он тогда воспользовался нумерацией по квадратам. Точно так же поступим и мы, только с тем осложнением, что некоторые дроби будем пропускать. Получится следующая нумерация положительных рациональных чисел:

\[
1, \hspace{1.5mm} 2, \hspace{1.5mm} {1 \over 2}, \hspace{1.5mm} 3, \hspace{1.5mm} {3 \over 2}, \hspace{1.5mm} {2 \over 3}, \hspace{1.5mm} {1 \over 3}, \hspace{1.5mm} ...
\]

Мы занумеровали, таким образом, все положительные рациональные числа. А теперь уже легко понять, как нумеруются все (то есть положительные и отрицательные) рациональные числа. Для этого надо записать их отдельно в виде двух таблиц и числа одной таблицы нумеровать четными номерами, а второй \(-\) нечетными (и еще оставить один номер для нуля).

\begin{theorem}
Объединение конечного или счётного числа конечных или счётных множеств конечно или счётно.
\end{theorem}

\begin{proof}
Воспользуемся приемом нумерации по квадратам. Если множеств конечное число или какие-то из множеств конечны, то в этой конструкции части членов не будет \(-\) и останется либо конечное, либо счётное множество.
\end{proof}

\begin{theorem}
Подмножество счётного множества конечно или счётно.
\end{theorem}

\begin{proof}
Пусть \(B\) \(-\) подмножество счётного множества \(A\). Представим множество \(A\) как последовательность \(\{a_0, a_1, a_2, ...\}\) и выбросим из неё те члены, которые не принадлежат \(B\) (сохраняя порядок оставшихся). Тогда оставшиеся члены образуют либо конечную последовательность (и тогда \(B\) конечно), либо бесконечную (и тогда \(B\) счётно).
\end{proof}

\begin{theorem}
Если каждый элемент множества \(A\) можно задать конечным набором натуральных чисел, то это множество конечно или счетно.
\end{theorem}

\begin{proof}
Способ нумерации основан на той же идее, с помощью которой пробовал решить свою самую трудную задачу директор гостиницы.

Именно, возьмем все простые числа и обозначим их \(p_1, p_2, ..., p_n, ...\). Если элемент \(x \in A\) задается набором натуральных чисел \(\{m_1, m_2, ..., m_n\}\), то поставим ему в соответствие натуральное число
\[
N_x = \prod_{i=1}^{n} {p_i}^{m_i}.
\]
Из теоремы об единственности разложения натуральных чисел на простые множители вытекает, что разным элементам из \(A\) при этом соответствуют разные натуральные числа. Поэтому отображение \(x \to N_x\) устанавливает взаимно однозначное соответствие между элементами множества \(A\) и частью множества натуральных чисел.

Из теоремы 3.2 следует, что каждое подмножество множества натуральных чисел конечно или счетно. Таким образом, множество \(A\) конечно или счетно.
\end{proof}

С помощью этой общей теоремы можно, например, доказать счетность множества всех алгебраических чисел.

\textit{Алгебраическими числами} называют корни алгебраических уравнений
\[
a_0x^n + a_1x^{n - 1} + ... + a_n = 0
\]
с целыми коэффициентами \(a_0, ..., a_n\). Неалгебраические числа называют \textit{трансцендентными}.

Каждое алгебраическое уравнение \(n\)-й степени имеет не более \(n\) корней. Поэтому каждое алгебраическое число задается набором чисел
\[
\{k, a_0, a_1, ..., a_n\},
\]
где \(a_0, ..., a_n\) \(-\) коэффициенты уравнения, a \(k\) \(-\) номер корня. Числа \(a_0, ..., a_n\) принимают всевозможные целые значения, a \(k\) \(-\) целые значения от 1 до \(n\). Применяя теорему, убеждаемся, что множество всех алгебраических чисел счетно. То обстоятельство, что числа \(a_0, ..., a_n\) принимают целые, а не только натуральные значения, несущественно, так как целые числа можно перенумеровать. \newline

Мы уже выяснили, что значат слова «два множества имеют поровну элементов». А теперь выясним, что значит «одно множество имеет больше элементов, чем второе».

Вспомним наш пример с танцплощадкой. Если после того, как заиграет оркестр и юноши пригласят девушек танцевать, некоторые нерасторопные юноши окажутся не у дел, то ясно, что юношей больше. Если же часть девушек будет с грустью наблюдать за своими танцующими подругами, то ясно, что больше девушек.

В этих случаях мы поступали так: устанавливали взаимно однозначное соответствие между одним множеством и частью другого множества. Если это удавалось, то отсюда следовало, что второе множество содержит больше элементов, чем первое.

К сожалению, для бесконечных множеств так просто поступить нельзя. Ведь мы уже видели, что множество может иметь столько же элементов, сколько и его часть. Поэтому только из того факта, что множество \(A\) имеет столько же элементов, сколько и часть множества \(B\), еще нельзя заключить, что оно имеет меньше элементов, чем все множество \(B\).

Мы будем скромнее в выражениях и скажем, что если множество \(A\) можно поставить во взаимно однозначное соответствие с некоторым подмножеством множества \(B\) (возможно, с самим \(B\)), то «множество \(B\) \textit{имеет не меньше элементов, чем} множество \(A\)» или «множество \(A\) \textit{по мощности не больше} множества \(B\)».

Следующие теоремы показывают, что это соотношение обладает всеми хорошими свойствами неравенств.

\begin{theorem}
Если \(A\) и \(B\) равномощны, то \(A\) имеет не большую мощность, чем \(B\). В частности, каждое множество \(A\) имеет не меньше элементов, чем само это множество.
\end{theorem}

\begin{proof}
Очевидно.
\end{proof}

\begin{theorem}
Если \(A\) имеет не большую мощность, чем \(B\), а \(B\) имеет не большую мощность, чем \(C\), то \(A\) имеет не большую мощность, чем \(C\).
\end{theorem}

\begin{proof}
Пусть \(A\) находится во взаимно однозначном соответствии с \(B' \subset B\), а \(B\) находится во взаимно однозначном соответствии с \(C' \subset C\). Тогда при втором соответствии \(B' \subset B\) соответствует некоторому множеству \(C'' \subset C'\) и потому \(A\) равномощно \(C''\).
\end{proof}

\begin{theorem}[Кантор, Бернштейн, Шрёдер]
\selectlanguage{french}
Soient \(X\), \(Y\) des ensembles déterminés, \(X_1\), \(Y_1\), des ensembles partiels de \(X\) et de \(Y\) respectivement. Nous devons démontrer que, étant \(X \sim Y_1\), et \(Y \sim X_1\), nous aurons toujours \(X \sim Y\).
\end{theorem}

\begin{proof}
La proposition \(X \sim Y_1\) signifie la supposition de la loi (I) suivante:

\textit{Un élément quelconque \(x\) de \(X\) détermine un et un seul élément \(y\) de \(Y\); donc cet \(y\) détermine aussi le \(x\) correspondant. Mais il y a un ou plusieurs éléments de \(Y\) qui ne figurent pas dans cette loi.}

De même la proposition \(Y \sim X_1\) signifie la supposition d'une loi (II), qu'il serait superflu de détailler encore.

Prenons donc un élément quelconque \(x_1\) de \(X\); après (I), il nous donne un élément déterminé \(y_1\) de \(Y_1 \subset Y\); cet élément \(y_1\) nous donne, puis par la loi (II), un élément déterminé \(x_2\) de \(X_1 \subset X\), etc. En faisant cela, nous ne \textit{comptons} pas; il n'y a là qu'un emploi des signes 1, 2, ... pour distinguer les éléments de \(X\).

Ainsi la suite
\[
x_{1}y_{1}x_{2}y_{2}...
\]
peut toujours être continuée à droite, mais pas toujours à gauche. Si \(x_1\) est un élément de \(X_1\), la loi (II) donne un élément \(y_0\), qui précède immédiatement \(x_1\); mais si \(x_1\) est un élément de \(X\), qui ne se trouve pas dans \(X_1\), la suite ne pourra plus être continuée à gauche.

On voit donc que les cas possibles sont les suivants:
\begin{enumerate}
    \item La suite commence avec un élément de \(X\).
    \item La suite commence avec un élément de \(Y\).
    \item La suite peut toujours être prolongée à gauche.
\end{enumerate}

Les éléments \(x_1^{'}\) et \(x_1^{''}\) de \(X\) nous donnent ainsi deux suites correspondantes:
\[
x_1^{'}y_1^{'}x_2^{'}y_2^{'}... \tag{1} \label{eq:eq1}
\]
\[
x_1^{''}y_1^{''}x_2^{''}y_2^{''}... \tag{2} \label{eq:eq2}
\]

S'il y a un élément commun dans les suites \eqref{eq:eq1} et \eqref{eq:eq2}, l'élément qui le suit est déterminé par la loi (I), en conséquence il sera le même dans les suites \eqref{eq:eq1} et \eqref{eq:eq2}, de même le précédent s'il y en a.

C'est-à-dire:

\textit{Un élément quelconque de \(X\) détermine toujours la suite correspondante.}

Il n'est pas nécessaire de détailler le cas spécial d'une suite \textit{périodique}. C'est évident, qu'une suite périodique peut toujours être prolongée à gauche.

La loi d'équivalence, dont l'expression est \(X \sim Y\), se trouve déterminée de fait par ces considérations.

Soit \(x\) un élément quelconque de \(X\); nous avons l'instruction pour la formation de la suite correspondante. Si cette suite commence avec un élément de \(X\), ou si elle peut être continuée à gauche, nous choisirons comme élément correspondant à \(x\) dans \(Y\) l'élément qui le suit dans la suite. Mais, si la suite commence avec un élément de \(Y\), nous prendrons comme élément correspondant dans \(Y\) celui qui précède \(x\) immédiatement. Ainsi l'équivalence \(X \sim Y\) est fixée.
\end{proof}

Теорема 3.6 значительно упрощает доказательства равномощности: например, если мы хотим доказать, что бублик и шар в пространстве равномощны, то достаточно заметить, что из бублика можно вырезать маленький шар (гомотетичный большому), а из шара \(-\) маленький бублик. \newline

Может случиться, что есть взаимно однозначное соответствие между множеством \(A\) и \(B_0 \subset B\), но не существует взаимно однозначного соответствия между \(A\) и всем множеством \(B\). Вот в этом случае мы и будем говорить, что \(B\) имеет больше элементов, чем \(A\). \newline

Мы уже говорили, что любая бесконечная часть множества натуральных чисел счетна. Это означает, что не может существовать бесконечное множество, мощность которого была бы меньше мощности счетного множества. Докажем теперь, что в каждом бесконечном множестве есть счетное подмножество. Отсюда будет следовать, что мощность счетного множества не больше мощности любого бесконечного множества, то есть что эта мощность \(-\) самая маленькая из бесконечных.

\begin{theorem}
Всякое бесконечное множество \(A\) содержит счётное подмножество.
\end{theorem}

\begin{proof}
Выберем из \(A\) один элемент \(x_1\) \(-\) это можно сделать, так как множество \(A\) бесконечно и, во всяком случае, не пусто. Ясно, что после удаления элемента \(x_1\) множество \(A\) не исчерпывается, и мы сможем выбрать из него второй элемент \(x_2\). После этого выберем третий элемент \(x_3\) и т. д. В результате мы извлечем из множества \(A\) счетное подмножество занумерованных элементов \(\{x_1, x_2, ..., x_n, ...\}\)
\footnote[1]
{
Немного усовершенствовав это доказательство, можно добиться, чтобы после удаления счетного подмножества осталось бесконечное множество. Для этого надо после извлечения подмножества вернуть обратно все элементы с четными номерами. В результате получится, что мы извлекли счетное подмножество \(\{x_1, x_3, x_5, ...\}\), а оставшееся множество еще содержит бесконечное множество элементов: \(\{x_2, x_4, x_6, ...\}\) (и, быть может, еще много других элементов).
}.
\end{proof}

Нетрудно доказать следующие теоремы.

\begin{theorem}
Мощность бесконечного множества не изменяется от прибавления к нему счетного множества.
\end{theorem}

\begin{theorem}
Мощность несчетного множества не меняется от удаления из него счетного множества.
\end{theorem}

Эти теоремы еще раз подтверждают, что счетные множества \(-\) самые малые из бесконечных множеств. \newline

Все построенные до сих пор множества оказались счетными. Это наводит на мысль, а не являются ли вообще все бесконечные множества счетными? Если бы это оказалось так, то жизнь математиков была бы легкой: все бесконечные множества имели бы поровну элементов и не понадобился бы никакой анализ бесконечности. Но выяснилось, что дело обстоит куда сложнее, несчетные множества существуют, и притом с разными мощностями.

Заметим, что доказать несчетность какого-то множества вообще нелегко. Ведь доказать, что какое-то множество счетно, это значит просто придумать правило, по которому нумеруются его элементы. А доказать несчетность какого-то множества, это значит доказать, что такого правила нет и быть не может. Иными словами, какое бы правило мы ни придумали, всегда найдется незанумерованный элемент множества. Чтобы доказывать несчетность множеств, Кантор придумал очень остроумный способ, получивший название диагонального процесса. Метод доказательства Кантора станет ясен из следующего рассказа Йона Тихого. \newline

До сих пор я рассказывал об удачах директора необыкновенной гостиницы: о том, как ему удалось вселить в заполненную гостиницу еще бесконечно много постояльцев, а потом даже жителей из бесконечного множества столь же необычных гостиниц. Но был случай, когда и этого мага и чародея постигла неудача.

Из треста космических гостиниц пришел приказ составить заранее все возможные варианты заполнения номеров. Эти варианты потребовали представить в виде таблицы, каждая строка которой изображала бы один из вариантов. При этом заполненные номера должны были изображаться единицами, а пустые нулями. Например вариант
\[
101010101010...
\]
означал, что все нечетные номера заняты, а все четные пустые, вариант
\[
11111111111...
\]
означал заполнение всей гостиницы, а вариант
\[
000000000000...
\]
означал полный финансовый крах \(-\) все номера пустовали.

Директор был перегружен работой и поэтому придумал простой выход из положения. Каждой дежурной по этажу было поручено составить столько вариантов заполнения, сколько номеров было в ее ведении. При этом были приняты меры, чтобы варианты не повторялись. Через несколько дней списки были представлены директору, и он объединил их в один список.

\(-\) Уверены ли вы, что этот список полон? \(-\) спросил я директора. \(-\) Не пропущен ли какой-нибудь вариант?

\(-\) Не знаю, \(-\) ответил он. \(-\) Вариантов в списке бесконечно много, и я не понимаю, как проверить, нет ли еще какого-нибудь варианта.

И тут у меня блеснула идея (впрочем, быть может, я несколько преувеличиваю свои способности, просто беседы с профессором Тарантогой о бесконечных множествах не прошли бесследно).

\(-\) Могу ручаться, что список неполон. Я берусь указать вариант, который наверняка пропущен.

\(-\) С тем, что список неполон, я еще соглашусь. А вот пропущенного варианта указать не удастся \(-\) ведь здесь уже бесконечно много вариантов.

Мы заключили пари. Чтобы выиграть его, я предложил прибить каждый вариант на дверь того номера, которому он соответствовал (если читатель помнит, вариантов было составлено именно столько, сколько было номеров в гостинице). А потом я поступил очень просто. Подойдя к двери первого номера, я увидел, что соответствующий вариант начинается с цифры 0. Немедленно в блокноте появилась цифра 1; это и была первая цифра варианта, который мне хотелось составить.

Когда я подошел к двери второго номера, то первая цифра соответствующего варианта меня не интересовала, ведь первая цифра моего варианта была уже написана. Поэтому все внимание было обращено на вторую цифру. Увидев, что эта цифра 1, я записал в своем блокноте цифру 0. Точно так же, обнаружив, что третья цифра варианта, прибитого к двери третьего номера, тоже 1, я записал в блокноте цифру 0. Вообще, если я обнаруживал, что \(n\)-я цифра \(n\)-го варианта есть 0, то писал в своем блокноте на \(n\)-м месте цифру 1, если же \(n\)-я цифра \(n\)-го варианта была 1, то я писал у себя 0.

Когда я обошел все номера гостиницы, то в блокноте оказалась записанной последовательность нулей и единиц.

Войдя в кабинет директора, я сказал:

\(-\) Вот, полюбуйтесь на пропущенный вариант.

\(-\) А откуда известно, что он пропущен?

\(-\) Он не может быть первым, так как отличается от него первой цифрой; не может быть вторым, так как отличается от него второй цифрой; третьим, так как отличается от него третьей цифрой; и вообще \(n\)-м, так как отличается от него \(n\)-й цифрой.

Пари было выиграно, и я получил вечное право бесплатного проживания в этой гостинице.

Но одновременно стало ясно, что какое бы счетное множество вариантов ни взять, всегда найдется вариант, не вошедший в это множество (эти варианты всегда можно развесить по дверям номеров). А это и значит, что множество всех вариантов заполнения гостиницы несчетно, задача, поставленная перед директором, оказалась невыполнимой.

Было решено дать об этом телеграмму. Надо сказать, что и телеграф в необыкновенной гостинице был тоже необычным, он передавал телеграммы, состоящие не из конечного, а из бесконечного (точнее говоря, счетного) множества точек и тире. Я сразу сообразил, что и множество таких телеграмм тоже несчетно, ведь вместо точек и тире можно ставить нули и единицы, а тогда не будет никакой разницы между телеграммами со счетным множеством знаков и множеством всех вариантов заполнения гостиницы.

Отправив телеграмму, я тепло попрощался с директором гостиницы и полетел в галактику РЩ-8067, где должен был произвести астрографическую съемку... \newline

Итак, доказана

\begin{theorem}
Множество бесконечных последовательностей нулей и единиц несчётно.
\end{theorem}

Для доказательства несчетности множества вещественных чисел, заключенных между нулем и единицей, нам понадобится следующая теорема.

\begin{theorem}
Отрезок \([0, 1]\) равномощен множеству всех бесконечных последовательностей нулей и единиц.
\end{theorem}

\begin{proof}
В самом деле, каждое число \(x \in [0, 1]\) записывается в виде бесконечной двоичной дроби. Первый знак этой дроби равен 0 или 1 в зависимости от того, попадает ли число \(x\) в левую или правую половину отрезка. Чтобы определить следующий знак, надо выбранную половину поделить снова пополам и посмотреть, куда попадёт \(x\), и т. д.

Это же соответствие можно описать в другую сторону: последовательности \(x_1x_2x_3...\) соответствует число, являющееся суммой ряда
\[
\sum_{i=1}^{\infty} {x_i \over 2^i} \leq \sum_{i=1}^{\infty} {1 \over 2^i} = 1.
\]

Описанное соответствие пока что не совсем взаимно однозначно: двоично-рациональные числа (дроби вида \(m \over 2^n\)) имеют два представления. Например, число \(3 \over 8\) можно записать как в виде 0,011000..., так и в виде 0,010111.... Соответствие станет взаимно однозначным, если отбросить дроби с единицей в периоде (кроме дроби 0,1111..., которую надо оставить). Но таких дробей счётное число, поэтому на мощность это не повлияет.
\end{proof}

Для доказательства несчетности множества \(\mathbb{R}\) нам понадобится следующая теорема.

\begin{theorem}
Множество \(\mathbb{R}\) эквивалентно множеству всех чисел в интервале \((0, 1)\).
\end{theorem}

\begin{proof}
Соответствие можно установить с помощью функции
\[
x \to {1 \over \pi}\arctan{x} + {1 \over 2}.
\]
\end{proof}

В течение долгого времени математики имели дело лишь с алгебраическими числами. Лишь ценой больших усилий французскому математику Лиувиллю удалось найти несколько чисел трансцендентных.

И вдруг оказалось, что алгебраические числа, которые встречаются на каждом шагу, на самом деле являются величайшей редкостью, а трансцендентные числа, которые так трудно строить, \(-\) обычным правилом.

В самом деле, мы уже видели, что алгебраические числа образуют лишь счетное множество. Множество же всех вещественных чисел, как мы только что обнаружили, несчетно. Значит, несчетна и разность множества вещественных чисел и множества алгебраических чисел, то есть множество трансцендентных чисел. \newline

С тем, что на бесконечной прямой столько же точек, сколько и на отрезке, математики, скрепя сердце, примирились. Но следующий результат Кантора оказался еще более неожиданным. В поисках множества, имеющего больше элементов, чем отрезок, он обратился к множеству точек квадрата. Сомнения в результате не было: ведь отрезок целиком размещается на одной стороне квадрата, а множество всех отрезков, на которые можно разложить квадрат, само имеет ту же мощность, что и множество точек отрезка.

На протяжении трех лет Кантор искал доказательство того, что взаимно однозначное соответствие между точками отрезка и точками квадрата невозможно.

Шли годы, а желанный результат не получался. И вдруг совершенно неожиданно ему удалось построить соответствие, которое он считал невозможным! Сначала он сам не поверил себе. Математику Дедекинду он писал: «Я вижу это, но не верю этому».

\begin{theorem}
Квадрат (со внутренностью) равномощен отрезку.
\end{theorem}

\begin{proof}
Квадрат равномощен множеству \([0, 1] \times [0, 1]\) пар вещественных чисел, каждое из которых лежит на отрезке \([0, 1]\) (метод координат). Мы уже знаем, что вместо чисел на отрезке можно говорить о последовательностях нулей и единиц. Осталось заметить, что паре последовательностей нулей и единиц \((x_0x_1x_2..., y_0y_1y_2...)\) можно поставить в соответствие последовательность-смесь \(x_0y_0x_1y_1x_2y_2...\) и что это соответствие будет взаимно однозначным.
\end{proof}

Рассматривая примеры, приведенные выше, можно заметить, что иногда бесконечное множество оказывается эквивалентным своей истинной части. Например, натуральных чисел оказывается «столько же», сколько и всех целых или даже всех рациональных; на интервале \((0, 1)\) «столько же» точек, сколько и на всей прямой, и т.д. Это явление характерно для бесконечных множеств.

\begin{theorem}
Всякое бесконечное множество эквивалентно некоторому своему собственному подмножеству.
\end{theorem}

\begin{proof}
Из всякого бесконечного множества \(M\) можно выбрать счетное подмножество (теорема 3.7); пусть \(A = \{a_1, ..., a_n, ...\}\) такое подмножество.

Разобьем его на два счетных подмножества \(A_1 = \{a_1, a_3, a_5, ...\}\) и \(A_2 = \{a_2, a_4, a_6, ...\}\) и установим между \(A\) и \(A_1\) взаимно однозначное соответствие. Это соответствие можно затем продолжить до взаимно однозначного соответствия между множествами \(A \cup (M \setminus A) = M\) и \(A_1 \cup (M \setminus A) = M \setminus A_2\), отнеся каждому элементу из \(M \setminus A\) сам этот элемент. Между тем множество \(M \setminus A_2\) не совпадает с \(M\), т. е. является собственным подмножеством для \(M\).
\end{proof}

Мы познакомились пока что с двумя типами бесконечных множеств. Одни из них имеют столько же элементов, сколько и множество натуральных чисел, а другие \(-\) столько же, сколько и множество точек на прямой. Оказалось, что во втором множестве больше элементов. Естественно, возникает вопрос, а нет ли «промежуточного» множества, которое имело бы больше элементов, чем множество натуральных чисел, и меньше, чем множество точек на прямой? Этот вопрос получил название \textit{проблемы континуума}
\footnote[1]
{
Вопрос оказался глубоко затрагивающим основания математики. Он был окончательно решен в 1963 г. американским математиком П. Коэном. Коэн доказал неразрешимость гипотезы континуума, показав, что и она сама, и ее отрицание порознь не противоречат принятой в теории множеств аксиоматике, а потому гипотеза континуума не может быть ни доказана, ни опровергнута в рамках этой аксиоматики, \(-\) ситуация, вполне аналогичная независимости пятого постулата Евклида о параллельных от остальных аксиом геометрии.
}. \newline

Пока что самой большой мощностью, которую мы знаем, является мощность множества точек на прямой, то есть мощность континуума. Ни множество точек квадрата, ни множество точек куба не имеют большей мощности. Не является ли мощность континуума самой большой? Оказывается, что нет. Более того, вообще нет множества самой большой мощности.

\begin{theorem}
Для любого множества \(A\) можно построить множество большей мощности.
\end{theorem}

\begin{proof}
Рассмотрим множество \(B\) всех функций, заданных на множестве \(A\) и принимающих значения 0 и 1.

Покажем, что мощность множества \(B\) не меньше, чем мощность множества \(A\). Для этого каждой точке \(a\) множества \(A\) поставим в соответствие функцию \(f_a\), принимающую в этой точке значение 1, а в остальных точках значение 0. Ясно, что разным точкам соответствуют разные функции.

Итак, мощность множества \(B\) не меньше мощности множества \(A\). Покажем теперь, что эти мощности не равны друг другу, то есть, что нет взаимно однозначного соответствия между элементами множеств \(A\) и \(B\). В самом деле, предположим, что такое соответствие существует.

Обозначим тогда функцию, соответствующую элементу \(a\) из \(A\), через \(\psi_a\). Напомним, что \(\forall x \in A \hspace{2mm} \psi_a(x) \in \{0, 1\}\).

Составим новую функцию \(\varphi\), заданную равенством
\[
\varphi(x) = 1 - \psi_x(x) \hspace{4mm} \forall x \in A.
\]

Ясно, что функция \(\varphi\) также задана на множестве \(A\) и принимает значения 0 и 1. Следовательно, \(\varphi \in B\). Но тогда, по предположению, \(\varphi\) соответствует некоторой точке \(\xi \in A\), а значит, \(\varphi = \psi_\xi\). Учитывая определение \(\varphi\), получаем, что для \(\forall x \in A\)
\[
\psi_\xi(x) = 1 - \psi_x(x).
\]

Положим в этом равенстве \(x = \xi\). Мы найдем тогда, что \(\psi_\xi(\xi) = 1 - \psi_\xi(\xi)\) и потому
\[
\psi_\xi(\xi) = {1 \over 2}.
\]

Но это противоречит тому, что значения функции \(\psi_\xi\) равны 0 и 1. Полученное противоречие показывает, что взаимно однозначного соответствия между множествами \(A\) и \(B\) быть не может.
\end{proof}

\end{document}\grid
