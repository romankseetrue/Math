\documentclass{article}

\usepackage[utf8]{inputenc}
\usepackage[russian]{babel}
\usepackage{amsthm}
\usepackage{amsmath}
\usepackage{mathtools}
\usepackage{amssymb}
\usepackage[symbol]{footmisc}

\title{Элементы теории множеств}
\author{Роман Кушниренко}

\makeatletter
\newcommand*{\rom}[1]{\expandafter\@slowromancap\romannumeral #1@}
\makeatother

\renewcommand{\thefootnote}{\fnsymbol{footnote}}

\newtheorem{theorem}{Теорема}[section]
\newtheorem*{lemma}{Лемма}
\newtheorem{claim}{Утверждение}[section]
\newtheorem{example}{Пример}
\newtheorem{definition}{Определение}[section]
\newtheorem*{consequence}{Следствие}

\begin{document}

	\maketitle

	\section{Множества и действия над ними}

В этой главе будет рассказано о том, что такое множества и какие действия можно выполнять над ними. К сожалению, основному понятию теории \(-\) понятию множества \(-\) нельзя дать строгого определения. Разумеется, можно сказать, что множество \(-\) это «совокупность», «собрание», «ансамбль», «коллекция», «семейство», «система», «класс» и т. д. Однако все это было бы не математическим определением, а скорее злоупотреблением словарным богатством русского языка. \newline

Возможны различные способы задания множества. Один из них состоит в том, что дается полный список элементов, входящих в множество. Но этот способ применим только к конечным множествам, да и то далеко не ко всем. Например, хотя множество всех рыб в океане и конечно, вряд ли его можно задать списком. А уж о составлении такого списка для бесконечного множества и думать нечего.

В тех случаях, когда множество нельзя задать при помощи списка, его задают путем указания некоторого характеристического свойства \(-\) такого свойства, что элементы множества им обладают, а все остальное на свете не обладает. \newline

В самых различных вопросах встречаются разбиения тех или иных множеств на попарно непересекающиеся подмножества (классы).

При разбиении множества на классы часто используют понятие \textit{эквивалентности} элементов. Для этого определяют, что значит «элемент \(x\) эквивалентен элементу \(y\)» (\(x \sim y\)), после чего объединяют эквивалентные элементы в один класс.

Однако не всякое понятие эквивалентности годится для такого разбиения. Например, назовем двух людей эквивалентными, если они знакомы друг с другом. Такое определение эквивалентности окажется неудачным. Ведь может случиться, что человек \(X\) знаком с человеком \(Y\), человек \(Y\) знаком с человеком \(Z\), а люди \(X\) и \(Z\) друг с другом незнакомы. Тогда нам придется сначала отнести в один класс людей \(X\) и \(Y\) (они друг с другом знакомы), потом в тот же класс включить и \(Z\) (он знаком с \(Y\)), и у нас в одном классе окажутся незнакомые друг с другом \(X\) и \(Z\).

Чтобы не было таких неприятностей, нужно, чтобы для понятия эквивалентности выполнялись следующие три условия:

\begin{enumerate}
  \item \(x \sim x\) для любого элемента \(x\) (рефлексивность).
  \item Если \(x \sim y\), то \(y \sim x\) (симметричность).
  \item Если \(x \sim y\) и \(y \sim z\), то \(x \sim z\) (транзитивность).
\end{enumerate}

\begin{theorem}
Пусть \(S\) \(-\) некоторое множество. Тогда выполнение этих условий необходимо и достаточно для того, чтобы \(S\) можно было разбить на классы.
\end{theorem}

\end{document}\grid
